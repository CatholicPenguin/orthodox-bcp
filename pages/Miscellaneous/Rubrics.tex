\phantomsection
\addcontentsline{toc}{section}{Rubrics}
\fancyhead[C]{{\LARGE Rubrics}}
\subby{Preface}
\lett{I}{n} the Christian life, the Holy Sacrifice of the Mass and the Daily Office serve as the highest modes of prayer. Due to their solemnity, it is important to attend carefully to their proper celebration.

	Through the Mass, the highest sacrifice is offered unto the Most Holy Trinity, and Christ is made substantially present upon the Altar. In the Office, the Christian obediently, faithfully, and joyfully observes the Sacred Scriptures' exhortation to offer the morning and evening sacrifice.
	
	The Christian who prays this Book, both layman and cleric, must keep the rubrics and their purpose within these prayers in mind. At the same time, however, the rubrics are to help the Christian offer his Sacrifice with loving care and faithfulness, not to bog him down with obligations. The Rites and Rubrics in this Book are laid out in a manner that should be accessible, especially after a few uses, lest `there be more business to find out what should be read, than to read it when it be found out.'
%It is in this mode of prayer that the Christian obediently, faithfully, and joyfully observes the Sacred Scripture's exhortation to offer the morning and evening sacrifice. And, through devotion to the Minor Hours, to `pray without ceasing'.\par
%It must be remembered, before a review of the rubrics, that the purpose of the Office is to facilitate this Sacrifice. Therefore, a simpler Office or an Office devoutly offered with imprecise rubrical distinctions, is immensely more pleasing than a despondent spirit.\par
%\emph{
%}
\subby{General Rubrics}
\begin{description}
\item[Amen] When the word \emph{Amen} is in italics, it should be said by the Congregation in response to the Minister's prayer.
%This modifies the 1928 rubric, in line with the 1549 and 1662, especially with consideration of Orthodox doctrine.
\item[Antecommunion] It may occur that a community lacks a Priest on Sunday or another day where the Service of Holy Communion is desired. In such a case, Mattins and the Great Litany should first be said. Then, the Holy Mass may be said from the \emph{Collect for Purity} until the Sermon, inclusive, concluding with the Additional Collects (p. \pageref{AdditionalCollects}).\par
\textsc{Note,} Mattins, the Great Litany, and Antecommunion should all be led by the Minister in the same place as the Office, not at the Altar.
%From the Monastic Diurnal
\item[Collects] When several proper Collects (or Secrets, or Postcommunions) are said, only the first and the last receive the full ending.\par
	\textsc{Note,} In the Mass, the \emph{Lord be with you} is only said to introduce the first and second collects.
\item[Commemorations] When multiple Feast Days overlap on the same day, the Office will be of the Feast Day of highest rank. The other Feast Day(s) will then be commemorated (or, in some cases, suppressed or transferred). Commemoration consists of praying the highest-ranking Feast Day's collect first followed by the lower ranking Feast Day(s)' collect. In the Mass, this also applies to the Secret and Postcommunion.
%Collect Conclusion taken from the English Missal 1940 (it's not provided in the Lay Missal).
%\item[Conclusion of Collects] When the Collect is addressed to God the Father, the ending is `Through Jesus Christ thy Son our Lord; who liveth and reigneth with thee in the unity of the Holy Spirit, one God, world without end. Amen.'\par
%If in the beginning of the Collect the Son be mentioned: `Through the same Jesus Christ thy Son our Lord; who liveth and reigneth with thee in the unity of the Holy Spirit, one God, world without end. Amen.'\par
%If at the end of the Collect the Son be mentioned: `Who with thee liveth and reigneth in the unity of the Holy Spirit, one God, world without end. Amen.'\par
%If the Collect be addressed to the Son: `Who livest and reignest with God the Father in the unity of the Holy Spirit, one God, world without end. Amen.'\par
%When the Holy Spirit is mentioned in the Collect: `In the unity of the same Holy Spirit, etc.'
%Taking what I can from the 1933 Lay Missal:
\item[Conclusion of Collects] When the Collect is addressed to God the Father, the ending is `Through Jesus Christ thy Son our Lord, who liveth and reigneth with thee, in the unity of the Holy Ghost, God, throughout all ages, world without end. Amen.'\par
If in the beginning of the Collect the Son be mentioned: `Through the same Jesus Christ thy Son our Lord; who liveth and reigneth with thee in the unity of the Holy Ghost, God, throughout all ages, world without end. Amen.'\par
If at the end of the Collect the Son be mentioned: `Who liveth and reigneth with thee in the unity of the Holy Ghost, God, throughout all ages, world without end. Amen.'\par
If the Collect be addressed to the Son: `Who livest and reignest with the Father, in the unity of the Holy Ghost, God, throughout all ages, world without end. Amen.'\par
If the Collect be addressed to the Holy Spirit: `Who livest and reignest with the Father and the Son, God, throughout all ages, world without end. Amen.'\par
When the Holy Spirit is mentioned in the Collect: `In the unity of the same Holy Ghost, etc.'
\item[Dominus Vobiscum] The Minister may only proclaim the \emph{Dominus vobiscum} (The Lord be with you) if he be in Major Orders. Otherwise, he must instead lead with the \emph{Domine, exaudi orationem nostram} (O Lord, hear our prayer) wherever the \emph{Dominus vobiscum} appears, or omit it entirely if it would be said twice consecutively.
\item[Ferial Commemorations] In Advent \& Lent, the propers for the First Sunday of Advent and Ash Wednesday may be said when commemoration of their season's respective Feria is indicated.
\item[Gloria Patri] During Passiontide, the \emph{Gloria Patri} is omitted in the Invitatory of Mattins, Responsory of the Minor Hours, Aspersion of the People, Introit, and Lavabo. During the Sacred Triduum, the \emph{Gloria Patri} is wholly omitted.
\item[Latin] In much of this Book, Latin is provided alongside the English. The Latin may always be used instead of the English, unless otherwise determined by the Ordinary. When Latin is not provided, but great devotion exists for the Latin language, the Latin which serves as the base text (such as the Latin BCP (1549 or 1552), Monastic Breviary, Missale, \& Rituale) may be used, with great concern for the catechesis and participation of the People.
\item[Liturgical Seasons] For the purpose of the rubrics, these festive seasons are referenced:
    \begin{itemize}
        \item Christmastide: From I Evensong of Christmas Day until I Evensong of Epiphany Day, exclusive.%\footnote{This is not to be confused with the `Christmas season', which can be said to end on Candlemas.}
        \item Epiphanytide: From I Evensong of Epiphany Day until I Evensong of Septuagesima Sunday, exclusive.
        \item Septuagesimatide (Pre-Lent): From I Evensong of Septuagesima Sunday until Mattins of Ash Wednesday, exclusive.
        \item Lent: From Mattins of Ash Wednesday until I Evensong of Passion Sunday, exclusive.
        \item Passiontide: From I Evensong of Passion Sunday until Mattins of Maundy Thursday, exclusive.
        \item Sacred Triduum: From Mattins of Maundy Thursday until I Evensong of Easter Sunday, exclusive.
        \item Eastertide (Paschaltide): From I Evensong of Easter Sunday until I Evensong of Trinity Sunday, exclusive.
        \item Trinitytide: From I Evensong of Trinity Sunday until I Evensong of the First Sunday of Advent, exclusive.
    \end{itemize}
\item[Minor Feast Days] Since only I and II Doubles are provided here in this Book (with some exceptions), the rubrics will be geared towards those. When the Office or Mass is celebrated with the rest of the Feast Days, the rubrics of the supplement used should be consulted.%\par
%The Mass propers and Office antiphons for the Feast Days not supplied in this Book may be found in the \textit{Book of Occasional Services}.\par
%For the rest of the Office propers, the Monastic Breviary should be consulted.

%Is this commentary really appropriate for the rubrics?:
% However, since these books have been published by Anglicans, great care must be taken to review the text in order to avoid any heterodox prayers.
\item[Mode of Recitation] Wherever it is indicated something be read, it may also be chanted, except for the Fore-Office and the Prayers after the Third Collect.
%\item[Plural \& Gendered Nouns and Pronouns] Sometimes there are words in the liturgy which change depending on number and gender, such as if the person being prayed for is a woman or if there are multiple people being prayed for. The Book, in accordance with English grammar, assumes the masculine singular, and if the prayer should be changed for either number or gender, the relevant words and letters are italicised.\par
%In cases where the change could only be due to number, such as prayers for Clerics (Subdeacons, Deacons, Priests, Bishops, etc.) which are---by divine law---always masculine, the relevant words are still italicised.
\end{description}

\subsec{Office Rubrics}
\begin{description}
%\item[Canticle Antiphons] %The antiphons for the \emph{Benedictus} and \emph{Magnificat} canticles are provided for those with special devotion to them.\par
%On Feast Days below Double, the antiphon may be said: as far as the {\dag} before, and repeated in full after, each Canticle.\par
%On Feast Days Double and higher, the antiphon may be said: entire before and after each Canticle.%\par
\item[Collects] The Collect to be said during the Office is provided in the relevant Proper. The Seasonal Prayers are to be said during Mass as indicated, and only their Collects may be said during the Office.
%The antiphons, in line with the Prayer Book Tradition, are optional.
\item[Ferial Days] The Ferial Office, i.e. the Simple Office of the occurrent Season, is always said on weekdays in Lent, on Ember Days, and on Rogation Monday, unless a I or II Double occur. In that case the Office is of the Feast with Commemoration of the Feria.\par
On weekdays in Advent and between Septuagesima and Ash Wednesday, and on Common Vigils, the Ferial Office is always said unless a Double Feast or Octave occur, and then Commemoration is made of such Ferias.\par
If a Simple Octave Day occur on one of these Ferias, it is only commemorated.\par
In like manner throughout the year the Office is of the Feria on those weekdays on which there does not occur a Double Feast, an Octave, or the Office of St. Mary on Saturday.
\item[Gloria Patri] After every Psalm and the indicated Canticles, except in the Office of the Dead or where otherwise indicated, the \emph{Gloria Patri} should be said.\par
It is customary to reverence the Trinity by bowing one's head during the \emph{Gloria Patri}, instead of crossing oneself.
%\item[Great Litany] The petition `From all sedition, privy conspiracy, and rebellion' may be replaced by the version from the 1549 Book of Common Prayer.
\item[Lectionary] The Lessons and Psalms for the days of the Church Year are provided in the lectionary, according to the Rankings of Feast Days (p. \pageref{ranking}).\par
\textsc{Note,} The proper psalms for Second Class and lower Sundays are optional.
\item[Octaves] The Office of an Octave is said (or Commemoration thereof made, when it is hindered by a Feast or a Sunday) through eight continuous days in Octaves of Feasts of the I Class. Octaves of Feasts of the II Class, which are Simple Octaves, are kept only on the Octave Day itself, with the rite of a Simple, unless hindered by a more worthy Office; but no notice is taken of the days within the Octave. If any Octave is not ended before Ash Wednesday, the Vigil of Pentecost, or 17 December, no notice is taken of it thenceforth.\par
A Simple Octave Day occurring within an Octave is only commemorated.
\item[Octave Days] On the Octave Day, except for the proper Psalms and Lessons, the Office is said as on the day of the Feast, unless otherwise noted. On a Simple Octave Day the Office is Simple.
\par
On Sundays within Privileged Octaves, the Office is said as directed in the Proper of Season. On Sundays within Common Octaves, the Office is of the Sunday with Commemoration of the Octave.
%\item[Octave Day Lessons] Only on the Feast Day are the proper Psalms and Lessons of the Feast Day used, not also on the Octave Day. During the Octave, the Collect for the Feast is used, according to relevant rules for Octave rankings.
\item[Office of the Dead] The Office of the Dead may be prayed in addition to the Daily Office. It may also be prayed instead of the Daily Office either when there is no Feast Day listed or when the Feast Day is of Memorial or Simple rank.\par
%CHECK:
The Office of the Dead may also replace the Daily Office of rank Double or lower on the day of death and the day of burial.
\item[Office of the Dead Antiphons] The Antiphons are not doubled except on 2 November; %14 November (at the Commemoration of All the Departed O.S.B., for Benedictines);
on the day of a burial; on the day after receiving tidings of a death; on the third, seventh, and thirtieth days; on the anniversary (even when transferred); and whenever the Office is celebrated solemnly.
\item[Office of the Dead Lessons] The lessons are always read without introduction or conclusion.
\item[Office of Our Lady on Saturday] On all Saturdays---except in Advent, Septuagesimatide, Lent, and on Ember Days---unless the Office be of a Double Feast (even transferred), or of an occurrent Octave or Vigil, or of a Sunday anticipated according to the rubrics, the Office is of St. Mary (p. \pageref{MarySaturday}).
%\item[Proper Psalms \& Lesson] Proper Lessons are given for every day of the Church Year and for every Feast Day supplied in this Book (Prayer Book Holydays). Proper Psalms are supplied for every Sunday, Ember Days, and Prayer Book Holydays.\par
%Only Feast Days of II Double or higher receive Proper Psalms and Lessons, unless otherwise noted or supplied in this Book.
\item[Scripture Verse Numbers] When a lesson or psalm is listed with only one verse (such as John 1:45 or Psalm 102:15), the Minister reads the that verse and the verses following for the rest of the chapter.
\item[Suffrages] The Suffrages in Prime and Compline are not said on Feast Days Double or higher.
\item[Sunday Collects] The collect for a Sunday is said on the I Evensong, Mattins, and II Evensong of the Sunday. It is also said throughout the week, unless a proper Collect is provided for the weekday. If the Office is Ferial, then the Sunday collect is read. %Otherwise, the collect for the Feast Day is read then the collect for the Sunday.
\item[Third Lesson] In accordance with pious custom, after the Second Canticle in the Major Hours, but before the Creed, it is fitting for a patristic lesson to be read from the Feast Day's Third Nocturn in Monastic or Sarum Matins.
\item[Vigils] %Vigils, unlike modern Roman practice, are not anticipated celebrations of the following day. Rather, they are days in their own right, dedicated to fasting in preparation for the Feast Day.\par
The Office of a Vigil is only said at Mattins.\par
If a Vigil fall on a Sunday (except for the Vigils of Nativity \& Epiphany), it is said or commemorated on the preceding Saturday.
\end{description}
\subby{Mass Rubrics}
\begin{description}
    \item[Chant] The Priest should be aware which parts of the liturgy, according to the solemnity of the Mass and the rank of the Feast Day, should be chanted.
    %\item[Commemorations] When Feast Days overlap, the highest ranking Feast Day is celebrated with commemoration of the lower ranking Feast Days, consisting of their Collect, Secret, and Postcommunion.
    \item[Conventual Masses] %The English Missal envisions that the Mass may be said in the context of a monastery, praying the full Monastic Breviary and having multiple Masses every day. Therefore, the conventual Mass is the higher Mass of the day, with large attendance, though the priests may celebrate a second Mass for a lower ranking Feast Day of that day, according to the rubrics.\par
    %Given that the Prayer Book Tradition does not envision many Masses being said in monastic fashion, rubrics from the English Missal regarding conventual and non-conventual Masses are largely irrelevant. For the ordinary parish church, it is safe to assume the day's Mass is a non-conventual Mass.
    Since this Book is oriented towards use in a parish church, it assumes the Mass being said is a `non-conventual Mass'.
    \item[Gloria in excelsis] The \emph{Gloria in excelsis} is said in the Mass on Feast Days of Double rank and higher. It is also said on all Sundays, except in Advent, Septuagesimatide, and Lent.\par
    For special rubrics regarding Votive Masses and Ferias, the \emph{Ordo} or Missal Rubrics should be consulted.
    \item[Introduction of the Epistle] The Epistle is introduced according to manner specified in the rubric, preferably using the title of the book as given in the 1611 King James Bible.
    \item[Introit] The Introit for each Mass is provided. The Introit is said as written, followed by the \emph{Gloria Patri} (except in Passiontide) and the Introit repeated up to the `\emph{Ps}' or `\emph{Cant.}'.
    \item[Manual Acts] For the manual acts of clerics during the Mass, consultation and familiarity should be made with \emph{Ritual Notes} and the English Missal, especially during Christmastide and Holy Week.
    %Rubric from English Missal:
    \item[Nicene Creed] The Nicene Creed is said after the Gospel on all Sundays throughout the year, even if the Mass is of a Feast on which it would not otherwise be said, or if the Sunday is vacant; from Christmas unto the Octave of St. John, inclusive; on the Feast of the Most Holy Name of Jesus; during the Epiphany Octave; Maundy Thursday; The Easter Octave; on the Ascension, Whitsunday, and Corpus Christi, and through their Octaves; on the Feast of the Heart of Jesus and through its Octave; on all Feasts of Blessed Mary and through their Octaves; on Feasts of Apostles and Evangelists and through their Octaves; on either Chair of St. Peter, and on St. Peter's Chains; on the Feasts of the Conversion and commemoration of St. Paul; on St. John before the Latin Gate; on the Feasts of St. Barnabas, the Invention and Exaltation of the Holy Cross; on the Feasts of the Precious Blood and the Transfiguration; on Feasts of Angels, St. Joseph, St. Mary Magdalene, and of the four Doctors, namely, Gregory, Ambrose, Augustine, and Jerome; in addition, on the Feasts of St. Hilary, St. Isidore, St. Leo I, the venerable Bede, and St. Peter Chrysologus; also on Feasts of the holy Doctors Athanasius, Basil, Cyril of Alexandria and Cyril of Jerusalem, Ephram Syrus, Gregory Nazianzen, John Chrysostom, and John Damascene; on the Octave-day of St. John Baptist, and of St. Laurence; on All Saints' Day, and through its Octave; on the Dedication of St. Saviour, and of the Holy Apostles Peter and Paul; on the anniversary of the dedication of a Church, and through its Octave; on the day of consecration of a Church, or of an Altar; on the Feasts of the Patron Saint of a Church, and on the day of a Saint where his Body or a considerable Relic is preserved; on the day of the creation and coronation of the Chief Bishop, and on the anniversary of such day; on the day, and the anniversary, of the election and consecration of a bishop.\par
    %Likewise on all Feasts which are celebrated on a Sunday or during Octaves, when it is said because of the Sunday and the Octave; likewise on the Feast of the Patron of any place, or the Title of a Church (not, however, of a Chapel or Altar); and on the chief Feasts of Monasteries and through their Octaves, in the Churches only of that Monastery.
    \par
    Also, the Creed is said in Votive Masses, which are celebrated solemnly for a grave matter or for a public Church cause, even if they are said in violet vestments on a Sunday.
    \item[Preface] The Prefaces with Solemn Chant are to be used in all Masses of any Double or Semidouble Office, and in Votive Masses for a grave and public cause. They are never used in Masses of Simple rite, or in Votive Masses which are not for a grave and public cause.\par
    \textsc{Note,} When a Priest is to use a Preface with a specified chant, he should have the text prepared beforehand and ready on or by the Altar.\par
        \begin{description}
            \item[Advent] The \nth{1}, \nth{2}, \nth{3}, and \nth{4} Sundays of Advent. The Vigil of the Nativity. On all Feasts, celebrated during Advent, which lack a proper preface.
            \item[Christmas] On Christmas Day and throughout the Octave in all Masses, even in those which would otherwise have a proper Preface---provided that there be a commemoration in them of the Octave or of the Sunday within the Octave, and the Mass itself or the commemoration do not demand another Preface of the Divine Mysteries or Persons.\par
            It is also said, according to the Missal, up to the Vigil of the Epiphany, inclusive.
            \item[Epiphany] From Epiphany Day until the Octave Day of the Epiphany, inclusive, as long as commemoration is made of the Octave or the Sunday within the Octave, unless the Mass (due to either the Feast Day or a Commemoration) requires another Preface.
            \item[Lent] From Ash Wednesday until the Saturday before Passion Sunday, inclusive, unless the Feast Day or first Commemoration refers to a Feast or Mystery of the Lord and requires another preface.
            \item[Holy Rood] From Passion Sunday until Maundy Thursday, inclusive. And all Masses of the Holy Rood, the Passion of the Lord, or the Most Precious Blood, even when celebrated within the Octave of the Nativity of Our Lord.\par
            Also, when Commemoration is made of Passiontide or of the aforesaid Mysteries, unless the Feast Day or the first Commemoration requires another Preface. And also, within the common Octave of any of the aforesaid Mysteries, even without a Commemoration.
            \item[Easter] From Holy Saturday until the Vigil of Ascension Thursday, inclusive, unless the Feast Day or first Commemoration requires another Preface.\par
            And also during Rogation Days and 25 April.
            \item[Ascension] From Ascension Thursday until the Friday after the Octave of the Ascension, inclusive, as long as commemoration is made of the Octave or of the Sunday within the Octave, unless the Feast Day or the first Commemoration does not require another Preface. It is always said, even without a Commemoration, on the Friday after the Octave.
            \item[Holy Ghost] From the Vigil of Whitsunday until Trinity Sunday, exclusive, unless the Feast Day or the first Commemoration requires another Preface, and in Votive Masses of the Holy Ghost, even when celebrated within the Octave of the Nativity of Our Lord, and in Masses where Commemoration is made of the Holy Ghost.
            \item[Trinity] On Sundays, from the \nth{3} Sunday after Trinity until the Sunday Next before Advent, inclusive, and from Septuagesima until Quinquagesima, inclusive, and in all Masses celebrated with a Commemoration of the Most Holy Trinity, as long as the Feast Day or first Commemoration does not require another Preface.\par
            Also, within the Octave of the Trinity, even without its Commemoration. And also in Masses of the Most Holy Trinity, even when celebrated within the Octave of the Nativity of the Lord.
            \item[Divine Compassion] Feast of the of the Compassion of Our Lord. Masses celebrated within the Octave of Divine Compassion---and Masses celebrated with a Commemoration of Divine Compassion---provided that a Commemoration is made of the Octave, and that the Mass itself, or the Commemoration first made, do not require another Preface.
            \item[Of Our Lord] Feasts of the Purification, Annunciation, and Transfiguration.
            \item[Blessed Virgin Mary] In Masses of the Blessed Virgin Mary, and in Masses which are celebrated with a Commemoration of the Blessed Virgin Mary, and do not refer to any Feast or Mystery of the Lord, provided that the Feast Day or the first Commemoration does not require another Preface; and, even apart from its Commemoration, within a Common Octave of the Blessed Virgin Mary.
            \item[St. Joseph] In Masses of St. Joseph, Spouse of the Blessed Virgin Mary, and in Masses, which are celebrated with Commemoration of St. Joseph, and which do not refer to any Feast or the Virgin Mother of God: and, provided that the Mass itself, or the first Commemoration, do not require another Preface; and it is said in the Octave of St. Joseph.
            \item[Apostles \& Evangelists] In Masses of the Apostles or Evangelists and in a Mass of the Election and Enthronement of the Patriarch and their anniversaries. It is said also, according to the Rubrics, in all Masses which are celebrated with a Commemoration of the Apostles or Evangelists themselves, or with a Commemoration of a Mass for the Patriarch as above mentioned and impeded on that day, and which are not of any Feast or Mystery of the Lord, provided that the Mass itself, or a Commemoration already made, do not demand another Preface.
            \item[Requiem] All Masses of the Dead.
            \item[Christ the King] Masses of Our Lord Jesus Christ the King, even when they are celebrated within the Octave of the Nativity of Our Lord. And in all Masses, which are celebrated with a Commemoration of Jesus Christ the King, provided that the Mass itself, or the Commemoration first made, do not require another Preface; and it is said, according to the same rule, even apart from its Commemoration, within the common Octave of Jesus Christ the King wherever celebrated.
            \item[All Saints] Feast of All Saints.
            \item[Common] When no proper Preface is said, the Priest proceeds immediately from \emph{Almighty, Everlasting God.} to \emph{Therefore, with Angels and Archangels}.
            \end{description}
    \item[Reciting the Propers] In the Ordinary of the Mass, the rubrics specify who should be reading/chanting the propers. However, when a less than ideal number of ministers are available, it is assumed that a cleric of higher rank can take the place of the lower ranking cleric. For example, in the absence of a Deacon, the Priest or Bishop reads the Gospel; of a Subdeacon, the Deacon or higher the Epistle; of a choir (that is, any suitable layman), any present cleric.
    %Translation of the following rubric is from the English Missal:
    \item[Requiem Mass] On the first day of each month (outside Advent, Lent, and Eastertide) not hindered by a Double or Semidouble office, the principal Mass is said generally for departed Priests, Benefactors, and others. But if on it there is a Simple Feast, or Feria with its proper Mass, or the Mass of the preceding Sunday is to be resumed after being hindered, and no other day occurs in the week on which it can be resumed: in Cathedral and Collegiate Churches two Masses may be said: one for the departed, the other of the Simple Feast or aforesaid Feria. But in churches which are not Cathedral or Collegiate let a Mass of the day be said with a general commemoration of the departed.\par
    Besides, on the Monday of each week in which the Office is of the Feria, the principal Mass may be for the departed. If, however, there be a proper Mass of the Feria, or of a Simple, or if the Mass of the preceding Sunday is to be resumed, as above, in the Mass let there be a commemoration (as has been said) of the departed.\par
    Lent, however, is excepted, and the whole of Eastertide, and throughout the year when the Office is Double or Semidouble, at which times the Conventual Mass is not for the departed (except on the day of the burial, or the anniversary of the departed) nor is a commemoration made for them.\par
    The \emph{Dies irae} is said on All Souls' Day and on the day of the burial in all sung Masses, as also in those which are said on privileged days mentioned above. However, in other Masses, it may either be recited or omitted at the pleasure of the celebrant.
    \item[Seasonal Prayers] Throughout the year, as indicated in the rubrics, additional Collects, Secrets, and Postcommunions are to be said, assigned for the different seasons even within Octaves or on Vigils. On Marian Feast Days (or Feast Days which include her commemoration), when the second seasonal collect would be of St. Mary, it is instead of the Holy Ghost.\par
    The propers chosen for the Collect should match for the Secret and Postcommunion.
    \item[Signs of the Cross] In the Mass, when {\ding{64}} appears, unless otherwise indicated or manifestly evident, it indicates the manual act of making the sign of the Cross.\par
    In the Canon, if there be a cross ({\ding{64}}) in reference to Bread or Body, it indicates a sign of the cross made over the Host. Likewise, if there be a cross ({\ding{64}}) in reference to Wine or Cup (Chalice), it indicates a sign of the cross made over the Cup (Chalice). Otherwise, it indicates a sign of the cross made over the Gifts generally, unless otherwise stated.\par
    When the rubrics tell the Priest to make the sign of the Cross on himself, unless otherwise indicated or manifestly evident, it indicates making the sign of the Cross from forehead to breast.
    \item[Votive Mass of Our Lady on Saturday] On Saturdays not hindered by a Double or Semidouble Feast, Octave, Vigil, Lenten Feria, or Ember Day, or by the office of a Sunday anticipated, Mass is said of our Lady according to the season, as given in the Missal.\par
    %In Advent, however, if the office of our Lady be not said on Saturday, nevertheless the principal Mass is of her, with commemoration of Advent, unless it be an Ember Day or Vigil, as above.
\end{description}
\newpage
%\phantomsection
%\addcontentsline{toc}{section}{Rankings of Days}\label{ranking}
%\fancyhead[C]{{\Large Ranking of Days}}
\section*{Ranking of Days}\label{ranking}
\noindent
\lett{F}{east} Days are ranked in order of priority. This is important, since it determines both what is prayed in the changeable parts of the Office (the Propers) and how to determine which Feast outranks the other(s) when Feast Days overlap on the same day.\par
The rankings for all of the Feast Days throughout the year are provided in the Kalendar (p. \pageref{kalendar}). However, there are some rankings which are not provided there or may need to be explained. For most people, it is expedient to order a yearly Ordo from the Antiochian Western Rite Vicariate which lays out the correct ordering for each day. However, to determine it without an Ordo, important rankings are described here.
\begin{itemize}
    \item Feast Days
        \begin{description}
            \item[First Class Double] (I Double)
            \item[Second Class Double] (II Double)
            \item[Greater Double]
            \item[Double] Feasts of Double or higher have I and II Evensong. A Double Feast begins with I Evensong and ends with Compline of the following day.
            \item[Semidouble] Sundays and days within Common Octaves are Semidouble.\par
            The Semidouble Office begins with I Evensong and ends with Compline of the following day.
            \item[Memorial] No Office is to be said for a Memorial, but on the day on which they are noted in the Kalendar, Commemoration only is made of them at I Evensong and Mattins; except on II Doubles, on which Commemoration of the Memorial is omitted at Evensong; and on I Doubles, on which no Commemoration is made of Memorials.
            \item[Simple] The Office is Simple on weekdays when the Office is of the Feria, and on the Octave Day of a Simple Octave.\par
                The Simple Office begins with I Evensong, and ends with None of the following day.
        \end{description}
    \item Ferial Days
        \begin{description}
            \item[Greater Privileged Feria] No Feast Day may be celebrated.
            %Not sure if this is true of the English Office:
                %The Office of the three Greater Ferias of Holy Week is said as directed in the Proper of the Season.
                \begin{itemize}
                    \item Ash Wednesday
                    \item Holy Monday
                    \item Holy Tuesday
                    \item Holy Wednesday
                \end{itemize}
            \item[Greater Feria] Begins at Mattins
                \begin{itemize}
                    \item Weekdays in Advent
                    \item Weekdays in Lent \& Passion Week
                    \item Ember Days
                    \item Rogation Monday
                \end{itemize}
            \item[Feria] Begins when the Office of the preceding day ceases. The Ferial Office ends at None if a Double or Simple follows.
        \end{description}
    \item Octaves
        \begin{itemize}
            \item Privileged Octaves
                \begin{description}
                    \item[First Order Privileged Octaves] The Feast Day, and second \& third days of the Octave, are I Double. The rest of the Octave until the Octave Day is Semidouble. The Octave Day is I Double. Feast Days cannot outrank the Days within the Octave, nor the Octave Day.\par
                    A I or II Double is transferred to the first unhindered day after the Octave; other Feasts are commemorated, except on the Feast Day and the two following days.
                        \begin{itemize}
                            \item Easter Octave
                            \item Whitsun Octave
                        \end{itemize}
                    \item[Second Order Privileged Octaves] The Feast Day is I Double. The Days within the Octave are Semidouble, being only outranked by I Double, with the Octave commemorated. The Octave Day is Greater Double, and gives place only to a I Double of universal observance
                        \begin{itemize}
                            \item Epiphany Octave
                            \item Corpus Christi Octave
                        \end{itemize}
                    \item[Third Order Privileged Octaves] Days within the Octave are trumped by any feast over Simple. Any Double Feast within this Octave has its Office with Commemoration of the Octave, but on the Octave Day of the Ascension only a I or II Double is kept, with Commemoration of the Octave.
                        \begin{itemize}
                            \item Christmas Octave
                            \item Ascension Octave
                            \item Divine Compassion Octave
                        \end{itemize}
                \end{description}
            \item Common Octaves\par
            The Days within the Octave are Semidouble. A Double Feast occurring within a Common Octave has its Office with Commemoration of the Octave, unless otherwise noted.\par
            \textsc{Note,} Common Octaves are not commemorated during Lent.
                \begin{itemize}
                    \item Conception of the BVM Octave
                    \item Assumption Octave
                    \item Nativity of St. John Baptist Octave
                    \item Solemnity of St. Joseph Octave
                    \item Sts. Peter \& Paul Octave
                    \item All Hallows Octave
                    \item The Octave of the principal patron saint of a church, cathedral, order, town, diocese, province, or nation. 
                \end{itemize}
            \item Simple Octaves\par
            The Feast Day is II Double. The Octave Day is Simple. The Days within the Octave are not commemorated.
                \begin{itemize}
                    \item St. Stephen Octave
                    \item St. John the Evangelist Octave
                    \item Holy Innocents Octave
                    \item St. Laurence Octave
                    \item Nativity of the B.V.M. Octave
                    \item An Octave of Secondary Patrons
                \end{itemize}
        \end{itemize}
    \item Sundays
        \begin{description}
            \item[First Class] Sundays of the First Class give place to no Feast. A I or II Double Feast Day falling on a Sunday of the First Class is transferred to the first unhindered day.\par
            \textsc{Note,} When I or II Double Feast Day falls on (or is transferred to) Monday, I Evensong is of the Feast Day with a commemoration of the Sunday. Otherwise, it is of the Sunday with a commemoration of the Feast Day.
                \begin{itemize}
                    \item First Sunday of Advent
                    \item Sundays in Lent \& Passiontide
                    \item Easter Sunday
                    \item Low Sunday
                    \item Pentecost
                \end{itemize}
            \item[Second Class] Sundays of the Second Class give place only to I Doubles, and the Sunday is commemorated.
                \begin{itemize}
                    \item Second Sunday of Advent
                    \item Third Sunday of Advent
                    \item Fourth Sunday of Advent
                    \item Septuagesima
                    \item Sexagesima
                    \item Quinquagesima
                \end{itemize}
            \item[All Other Sundays] On lesser Sundays, the Office is of the Sunday unless a Double I or II Class occur thereon, when the Office is of the Feast with Commemoration of the Sunday.
        \end{description}
    \item Privileged Vigils\par
        \begin{description}
            \item[First Class] Vigils of the First Class are preferred over any other Feast Day.
                \begin{itemize}
                    \item Vigil of the Nativity of Our Lord
                    \item Vigil of Whitsunday
                \end{itemize}
            \item[Second Class] Vigils of the Second Class are preferred over any Feast Day, except I \& II Doubles and Feasts of Our Lord.
                \begin{itemize}
                    \item Vigil of Epiphany
                \end{itemize}
        \end{description}
    \item Common Vigils\par
        Non-privileged Vigils are preferred over any Feast Day, except Doubles \& above or an Octave.
\end{itemize}

%\subby{First \& Second Evensong}
%When I Evensong of a Feast Day overlaps with II Evensong of another Feast Day, one will always take priority.
%
%  	\begin{description}
 % 		\item[First Class Double with a Common Octave] I Evensong is always said.
 % 		\item[First Class Double] I Evensong is said, unless the II Evensong is of a First Class Double with a Common Octave. If the II Evensong Feast Day be of Greater Double or higher, Commemoration is made of it.
%  		\item[Second Class Double] I Evensong is said, unless the II Evensong be of a Second Class Double.\par
%  	\end{description}
