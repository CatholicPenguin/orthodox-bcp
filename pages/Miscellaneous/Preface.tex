\phantomsection
\addcontentsline{toc}{section}{Preface}
\fancyhead[C]{{\Large Preface}}
\noindent
Those coming into Holy Orthodoxy from the Anglican tradition bring with them a deep piety, a strong spiritual tradition, and a holy zeal. These are brought to the Church not as leverage, licence, or loftiness but as a gift. The wisdom of our fathers in the Orthodox faith has been to receive this gift with joy. To heal it of its brokenness, to make whole what is incomplete, and to unflinchingly retain what is good. This manifests the apostolic zeal of the Orthodox Church to receive and heal all men in their entirety: their selves---their souls and bodies---together with their liturgical tradition (in this case, the Prayer Book Tradition).\par
\subby{The Purpose of the Book of Common Prayer}\par\noindent
The Book of Common Prayer is the centrepiece of the English spiritual life: containing the highest form of prayer---the Holy Sacrifice of the Mass---together with the daily sacrifice of the Christian---the Daily Office---with other holy devotionals, sacraments, and rites. Countless Christians can testify to the spiritual fruit wrought by this unity of church and home and to the entrance of the rhythm of the Church Year into the hours of ordinary life. The beauty, benefit, and goal of this Book of Common Prayer is to have a uniform text which presents what it means to be an Orthodox Christian in the English expression of the Western Rite. It presents what we believe and how we live by our prayer.\par
The Synod of Russia saw the wisdom of bringing this beautiful work into Holy Orthodoxy in its \textit{Observations}. After analysing the Book of Common Prayer and giving its observations of what would need to be changed, it encouraged American Anglicans to compile an Orthodox Book of Common Prayer. This present work, then, is simply a far too long overdue act of obedience to our fathers with a full confidence in their promise.\par
Unfortunately, in the meantime, the state of the English tradition in the Antiochian Western Rite Vicariate has experienced confusion. There is currently an unofficial Book of Common Prayer which is in wide circulation but without any approbation and not fully following the \textit{Observations}. There is a commonly used Office book with only bits and pieces of the Prayer Book Tradition. And the liturgy so far approved has unclear rubrics and a messy compilation of sources and preferences.\par
This Book of Common Prayer, therefore, seeks to fulfil the desire of our fathers in Christ by creating a standard text, prudently edited and to be made easily and widely available. The vision is to present both to Orthodox and to Anglicans considering union with the Church a single text which clearly presents the English expression of Catholic Orthodoxy. In doing this, we, as Orthodox in the Western Rite, follow our sainted fathers. We need only to look to St. John Maximovitch, who in his love for the French people (their persons and their rites and their traditions) blessed the use of the Gallican liturgy. Or we can turn our minds to St. Tikhon of Moscow who, out of love for the Americans under his care, worked for the reception of American Anglicans (their persons and their rites and their traditions) into the Orthodox Church.\par
And so we, trusting and following the wisdom of our fathers, present to the Antiochian Western Rite Vicariate an unwavering commitment to the Orthodox Faith and to every legitimate element of our liturgical heritage. It is the faithful fulfilment of the Orthodox missionary ethos and a firm testament to her commitment to such. And it serves a necessary role in the Vicariate.

\subby{Moving Forward}\par\noindent
We are beyond joyful to present this draft copy of the Book of Common Prayer unto thee, our Lord Bishop. We are thankful for thy sobermindness and fidelity to the mission of the Western Rite. We are hopeful that this Book of Common Prayer will be well-received. We pray that we can begin a period of discussion and review so that we may finally enjoy a Book of Common Prayer fitting both for our situation in the Vicariate and for the determined missionary zeal of the Orthodox Church. And we request and look forward to, after sufficient time for reflection and review, authorisation and distribution of this Book of Common Prayer for the benefit of the churches under thy care.

\begin{FlushRight}

To the Glory of God Alone \& by the Intercession of St. Tikhon of Moscow: Humbly Thine in Christ,

Augustine Watson
\end{FlushRight}