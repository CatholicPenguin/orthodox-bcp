\phantomsection
\addcontentsline{toc}{section}{Editor's Notes}
\fancyhead[C]{{\Large Editor's Notes}}
\noindent
The work of this Book of Common Prayer (BCP) has come from a need for an accessible Book of Common Prayer in the Antiochian Western Rite Vicariate (AWRV) as its liturgical forms have become more determined and concrete.
%This Prayer Book is decidedly `Common', not only in seeking approbation from our bishop but also in that it includes the text for both the Liturgies of St. Gregory and St. Tikhon in both its ordinary and propers. 

The goal of this Prayer Book is not to innovate but, following the Synod of Russia's \textit{Observations} and especially our own jurisdiction's \textit{Western Rite Edict and Directory}---along with reflection on the so-far approved liturgical books---to create a true successor to the 1928 Book of Common Prayer that is fully Orthodox and fitting for use in the AWRV.

\subsection{Prayer Book Sources of this Book of Common Prayer}\par\noindent
This BCP is decidedly an American Prayer Book. The base text is the 1928 BCP. This is not because the 1928 BCP is the best BCP, since certain elements entered the 1928 American edition, which were less than desirable or sometimes simply absurd. Rather, the AWRV shines in taking up communities and traditions where they are. While it must be appreciated that St. Tikhon's---and his Synod's---work with the American Prayer Book dealt with the 1892 edition, the unique position of the 1928 BCP---being the latest version in the authentic American Prayer Book tradition---lies in that most Anglicans who have become Orthodox, or who are considering Orthodoxy, either used the 1928 BCP or conformed their liturgical text to be more like it. In contrast, the 1892 BCP has no role in the life of any American parish, except as a reference text in a library.

Another difficult element to balance is our brethren in Canada. Given that it will be some time until the AWRV grows in Canada to the point of developing its own BCP, it seems meet to consult the 1918 and 1962 Canadian Books of Common Prayer for this edition. For example, when multiple options are allowed, such as in the Opening Sentences and the Prayers, effort is taken to include the Canadian tradition. The Canadian tradition is also more fitting and helpful in certain areas. For example, the American tradition excised the Athanasian Creed, so it is taken from the Canadian Prayer Book Tradition for this BCP.

Great care has been taken to keep the American Prayer Book Tradition intact, while carefully adjusting or removing anything contrary to Orthodox faith and practice. The most common problem which arises is deviation from the Latin text and the Prayer Book Tradition in order to obscure God's just wrath, man's sinfulness, his need for repentance, or the doctrine of the Trinity. In these cases, edits needed to be made, always according to prior tradition, not by engaging in original composition. In these cases, the Canadian (1918 and 1962) and English (1549 and Proposed 1928) Prayer Books were consulted.

This problem in the American Prayer Book Tradition also creates another concern in the context of the AWRV. In the AWRV, for the Roman tradition, the Monastic Diurnal and Monastic Matins are used, which often use the text of the 1662 BCP for its beautiful English translation of the Latin. Many of these psalms, canticles, and prayers, however, were altered in the American tradition. While the English and Roman traditions are each worthy and legitimate in their own rights with their own identities, it did not seem prudent to deviate from the translation used by our brethren using the Monastic Office, when their translation is from the Prayer Book Tradition. Therefore, the Psalter and Te Deum used in this BCP are from the 1662 Book of Common Prayer with no alterations.

While we understand these edits and adjustments involves a measure of subjectivity, we truly believe this BCP remains a rightful and faithful inheritor of the American Prayer Book Tradition, and we ask for forgiveness where we may have made any misstep.

\subsection{Material beyond the Prayer Book Tradition}\par\noindent
The main source for the extra-Prayer Book text is the 1933 English Missal, in the lay edition. The English Missal, as opposed to any other Anglican missal, is preferred since it has for some time been the missal used by the Antiochian Western Rite Vicariate, even though the American Missal has been used as well.

Of course, as is the signature of Knott, every edition (not just within the same year but even within the same version---lay or altar) has variations in the translation, so that no two 1933 or 1958 `English Missal's are the same. Therefore, there can be no work \textit{simply} based on the 1933 English Missal or 1958 English Missal. Rather, what is present here is from the copy the editor had on hand.

For this project, the 1933 English Missal is superior for three main reasons. First, among the English Missal traditions, the 1933 remains the most true to the underlying Latin and the true English character of the Anglican missal tradition, as opposed to later editions which reveal a growing desire of novelty and innovation. Second, while it seems that Knott never registered its copyright for its Missal in these United States, and the content found in it can be found in various earlier translations and missals, the 1933 English Missal will certainly pass into the public domain in 2029, avoiding any possible doubts or concerns about intellectual property. Third, the 1933 English Missal is the most easily accessible in these United States. The English Missal, in general, is very difficult to access as an American, and later editions are nearly impossible to access, never mind own.

%While we tried to avoid anything special, unique, or `trend-setting', we do hope (by necessity) that this ends up serving as the definitive text of the Knott tradition.

For the parts of the Daily Office which come from the Breviary, we followed the 1932 Monastic Diurnal for similar reasons, though with far fewer complications than Knott's publication.

For some elements of the Daily Office and the Mass which are not found in either the English Missal or the Monastic Diurnal, we used the 1868 translation of the Sarum Missal and the 1874 translation of the Sarum hours. While this BCP is not a Sarum production, certain elements of the Prayer Book Tradition require reference to the Sarum Rite for a fuller expression. For example, the Opening Sentences come from the sacerdotal versicles which fell between Matins and Lauds. Therefore, for a fuller set of Opening Sentences for the various saints and seasons, we used the Sarum versicles, except for a few Feast Days absent in the Sarum Rite. Also, since care has been taken to provide both the English and Roman propers for Feast Days, where the Prayer Book Tradition lacked propers for a saint, and when the Sarum differed greatly from the Roman, the Sarum propers were provided. In one instance, however, the Sarum alone was provided: the Feast of the Conception of the Blessed Virgin Mary. In this case, the Roman propers for this Feast contain both liturgical and theological novelties since the dogmatisation by Rome of the Immaculate Conception. Since the English Missal does not provide the older Roman propers, the Sarum was simply provided.

\subsection{Selection of Feast Days}\par\noindent
The beauty of the Prayer Book Tradition is its noble simplicity. It provides an accessible liturgical form for parochial use. Because this BCP is in line with this tradition, and does not intend to be a Missal or Rituale, it seemed best to only provide propers for Feast Days of the rank Second Double and greater, except for some lower ranking Feast Days which are present in the Prayer Book Tradition (though, every Feast Day with its rank is provided in the Kalendar). We hope, if this BCP is well-received, to later publish a Book of Occasional Services containing the rest of the propers and the Holy Week services.

We hope that this Book of Common Prayer advances the mission of the Orthodox Catholic Church in its Western Rite and benefits all men of good will who find benefit from it.