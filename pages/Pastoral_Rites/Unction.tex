%The Rite of Unction comes from the Orthodox Ritual
\fancyhead[RO,LE]{\textit{Holy Unction}}
\fancyhead[RE,LO]{\textit{}}
\section{Sacrament of Holy Unction}
\subby{Exhortation before Holy Unction}
\begin{secrubric}
    The Priest may, before administering Holy Unction, exhort the sick person with these words,
\end{secrubric}
%From the Rituale Romanum's English translation, the `He' changed to `he' and `Extreme Unction' to `Holy Unction'.
\lett{O}{ur} Lord and Saviour Jesus Christ has been pleased to institute, besides the Holy Communion, another heavenly medicine for the benefit of the sick, which is the Sacrament of %word change:
Holy Unction; according to what we read in the Epistle of St. James, where it is said: Is any man sick among you, let him bring in the priests of the Church, and let them pray over him, anointing him with oil in the name of the Lord: and the prayer of faith shall save the sick man; and the Lord shall raise him up, and if he be in sins, they shall be forgiven him (v. 14). --- You see here the authority for administering to the sick this holy Unction, from the express word of God. And also the great graces and benefits which God promises to bestow on every one who receives this Sacrament with proper dispositions, namely, that he will save the sick man, will raise him up from his sick bed, if he sees his recovery expedient for the welfare of his soul; and, what is infinitely more desirable than corporal health, will impart to him the forgiveness of his sins. Acknowledge, then, the infinite goodness of your Redeemer, and with the most lively sentiments of gratitude and love, embrace the great grace which is here prepared for you in this heavenly institution: and join your attention and devotion, with the prayers we shall now make to our Lord, for the healing of your soul and body, and to obtain for you the full remission of all your sins.\par
And as the eyes, the ears, and the other organs of sense, are the instruments by which men are led to offend Almighty God, they will on that account be anointed with the holy Oil: whilst we apply this holy Oil to your eyes, your ears, and your other senses, do you, with a contrite and humble heart, implore the mercy of God for the forgiveness of all the sins which through these avenues have made their way into your soul. Pray also for his supporting grace in this your illness, and that you may continue to the end ever faithful to him.
\subby{Devotion during Holy Unction}
\begin{rubric}
    While the Priest is administering this Sacrament to the sick person, one of the assistants may, before each Unction, read one of the following short prayers, corresponding to the organ of sense that is next to be anointed, that it may be repeated by the sick person.
\end{rubric}
My eyes have seen vanities, but now let them be shut to the world, and open to thee alone, my Jesus; and pardon me all the sins I have committed by my seeing.\par
My ears have been open to detraction, profaneness, and unprofitable discourses: let me now give ear to thy word, to thy commands, and thy call; and pardon me, O Jesus, all the sins I have committed by my hearing.\par
I have taken delight in the perfumes of this world, which are nothing but corruptions: now let my heart and prayers ascend like incense in thy sight, and pardon me, O Lord, all the sins I have committed by my sense of smell.\par
My tongue has many ways offended, both in speaking and tasting; now let its whole business be to cry for mercy: pardon me, dear Jesus, all the sins I have committed by words, or by any excess in eating and drinking.\par
My hands have offended in contributing to many follies injurious to myself and my neighbour: now let them be lifted up to Heaven in testimony of a penitent heart; and pardon me, O Lord, all the sins I have committed by the ill use of my hands.\par
My feet have gone astray in the paths of vanity and sin: now let me walk in the way of thy commandments; and forgive me, O Lord, all the sins I have committed by my disordered steps.
\subby{Unction of the Sick}
\begin{rubric}
    In case of necessity, it suffices to anoint one sense only, or more fittingly the forehead, with this shorter form:
\end{rubric}
\noindent
Through this holy {\ding{64}} Unction the Lord pardon thee whatsoever thou hast done amiss. Amen.
\begin{rubric}
    But the obligation remains, when the danger ceases, of supplying each of the anointings and all the prayers. If there be doubt whether the sick person yet lives, the Priest shall anoint conditionally saying:
\end{rubric}\par\noindent
If thou livest, through this holy {\ding{64}} Unction, etc.
\begin{rubric}
    The Priest, who is to administer the Sacrament of Holy Unction, shall take care that there be made ready, as far as possible, in the sick person's house, a table covered with a white cloth, and a vessel, wherein is cotton-wool or the like, divided into six small pellets, for wiping the places anointed; bread crumbs for cleansing the fingers; water for washing the Priest's hands, and a wax candle, to give him light as he anoints.
\end{rubric}
\begin{rubric}
    Then with the Clergy or at least with one Clerk, who carries a Cross, the holy Water with sprinkler, and the Ritual, the Parish Priest himself takes the vessel of sacred Oil, enclosed in a bag of violet silk, and carries it with care; for greater safety he may hang it around his neck. No bell is rung.
\end{rubric}
\begin{rubric}
    When he comes to the place where the sick person lies, the Priest entering the room, says:
\end{rubric}

\elcol{℣. Peace be to this house.}{℣. Pax huic dómui.}

\elcol{℟. And to all that dwell in it.}{℟. Et ómnibus habitántibus in ea.}

\begin{rubric}
    Having placed the Oil upon the table and (if not already vested) having put on surplice and violet stole, the Priest gives the sick person a Cross to kiss. Then he shall sprinkle him, the bystanders, and the room, in the form of a Cross, saying:
\end{rubric}\par\noindent
Thou shalt purge me, O Lord, with hyssop, and I shall be clean: thou shalt wash me, and I shall be whiter than snow.
\begin{rubric}
    If the sick person wishes to confess, he shall hear him and absolve him. Then he shall comfort him, and, if time permit, briefly admonish him concerning the power and efficacy of this Sacrament.
\end{rubric}

\elcol{℣. Our help {\ding{64}} is in the name of the Lord.}{℣. Adjutórium {\ding{64}} nostrum in nómine Dómini.}

\elcol{℟. Who hath made heaven and earth.}{℟.  Qui fecit c{\oe}lum et terram.}

\elcol{℣. The Lord be with you.}{℣. Dóminus vobíscum.}

\elcol{℟. And with thy spirit.}{℟. Et cum spíritu tuo.}
\par\noindent
\centerline{\textit{Let us pray.}}

\lett{L}{et} there enter into this house, O Lord, Jesu Christ, with the coming of us unworthy, everlasting happiness, divine prosperity, serene gladness, fruitful charity, and everlasting health: let no evil spirits approach this place: may Angels of peace be present, and all evil discord far removed from this house: magnify upon us thy holy Name, O Lord, and {\ding{64}} bless our conversation; sanctify our unworthy coming, who art holy and gracious, and abidest with the Father and the Holy Ghost, world without end. \textit{Amen.}\par
\lett{L}{et} us pray and beseech our Lord Jesus Christ, that in blessing he may {\ding{64}} bless this dwelling-place, and all that inhabit it, and give to them a good Angel guardian, and make them so to serve him, that they may consider the wondrous things of his law: may he turn from them all adverse powers: deliver them from all fear, and from every disquiet, and vouchsafe to preserve them in safety in this dwelling-place: Who with the Father and the Holy Ghost liveth and reigneth, world without end. \textit{Amen.}\par
\letuspray
\lett{G}{raciously} hear us, O Lord holy, Father almighty, everlasting God: and vouchsafe to send from heaven thy holy Angel to guard and cherish, protect, visit, and defend all who dwell in this place. Through Christ our Lord. \textit{Amen.}
\begin{rubric}
    These prayers, if time does not allow, may be omitted in whole or in part. Then the General Confession is made in the accustomed manner (p. \pageref{Confiteor}), and the Priest says in the singular number:
\end{rubric}
\elcol{℟. God Almighty have mercy upon thee, forgive thee thy sins, and bring thee to everlasting life.\par
℟. Amen.}{℟. Misereátur tui omnípotens Deus, et, dimíssis peccátis tuis, perdúcat te ad vitam {\ae}térnam.\par
℟. Amen.}
\elcol{℣. The Almighty and merciful Lord grant thee pardon, {\ding{64}} absolution, and remission of thy sins.\par
℟. Amen.}{℣. Indulgéntiam, {\ding{64}} absolutiónem et remissiónem peccatórum tuórum tríbuat tibi omnípotens et miséricors Dóminus.\par
℟. Amen.}
\begin{rubric}
    The Priest, extending his right hand over the head of the sick person, says:
\end{rubric}\par\noindent
In the Name of the {\ding{64}} Father, and of the {\ding{64}} Son, and of the {\ding{64}} Holy Ghost, may there be extinguished in thee all power of the devil, through the imposition of our hands, and through the invocation of the glorious and holy Virgin Mary Mother of God, and of her illustrious Spouse Joseph, and of all the holy Angels, Archangels, Patriarchs, Prophets, Apostles, Martyrs, Confessors and Virgins, and of all the Saints. \textit{Amen.}
\subby{Anointings}
\begin{rubric}
    Then, dipping his thumb in the holy Oil, he anoints the sick person in the form of a Cross on the parts mentioned below, adapting the words of the form to the appropriate place in this manner,
\end{rubric}
\begin{rubric}
    The Clerk, if he be in Holy Orders, or the Priest himself, after each anointing shall wipe the place anointed with a fresh pellet of cotton-wool, and place it in a clean vessel, and afterwards carry it to the Church, burn it, and cast the ashes into the sacrarium.
\end{rubric}
\begin{rubric}
    During the anointings, the bystanders should say suitable prayers for the sick person, e.g., one or more of the Penitential Psalms (6, 32, 38, 51, 102, 130, 143) or the Litany of the Saints (p. \pageref{LitanySaints}).
\end{rubric}
\subbysub{At the eyes}\noindent
Through this holy {\ding{64}} Unction, and his most tender mercy, the Lord pardon thee whatsoever thou hast done amiss by seeing. Amen.
\subbysub{At the ears}\noindent
Through this holy {\ding{64}} Unction, and his most tender mercy, the Lord pardon thee whatsoever thou hast done amiss by hearing. Amen.
\subbysub{At the nostrils}\noindent
Through this holy {\ding{64}} Unction, and his most tender mercy, the Lord pardon thee whatsoever thou hast done amiss by smelling. Amen.
\subbysub{At the mouth, the lips being closed}\noindent
Through this holy {\ding{64}} Unction, and his most tender mercy, the Lord pardon thee whatsoever thou hast done amiss by tasting and speaking. Amen.
\subbysub{At the hands}
\begin{rubric}
    \textsc{Note,} The hands of Priests are not anointed on the inside, but on the outside.
\end{rubric}\par\noindent
Through this holy {\ding{64}} Unction, and his most tender mercy, the Lord pardon thee whatsoever thou hast done amiss by touching. Amen.
\subbysub{At the feet}
\begin{rubric}
    \textsc{Note,} This unction of the feet may be omitted for any reasonable cause.
\end{rubric}\par\noindent
Through this holy {\ding{64}} Unction, and his most tender mercy, the Lord pardon thee whatsoever thou hast done amiss by walking. Amen.
\begin{rubric}
    All these things being done, the Priest rubs his thumb with bread crumbs, washes his hands, and wipes them with the towel. The water and bread shall in due course be cast into the sacrarium or the fire. Then he says:
\end{rubric}
\elcol{℣. Lord, have mercy upon us.}{℣. Kýrie, eléison.}
\elcol{℟. Christ, have mercy upon us.}{℟. Christe, eléison.}
\elcol{℣. Lord, have mercy upon us.}{℣. Kýrie, eléison.}
\subby{Conclusion}
\begin{rubric}
    The Lord's Prayer is said secretly until,
\end{rubric}\par
\elcol{℣. And lead us not into temptation;}{℣. Et ne nos indúcas in tentatiónem.}

\elcol{℟. But deliver us from evil.}{℟. Sed líbera nos a malo.}

\elcol{℣. O Lord, save thy \textit{servant}.}{℣. Salv\textit{um} fac \textit{servum tuum}.}

\elcol{℟. Who putteth \textit{his} trust in thee.}{℟. Deus meus, sperántem in te.}

\elcol{℣. Send \textit{him} help from thy holy place.}{℣. Mitte ei, Dómine, auxílium de sancto.}

\elcol{℟. And evermore mightily defend \textit{him}.}{℟. Et de Sion tuére \textit{eum}.}

\elcol{℣. Be unto \textit{him}, O Lord, a tower of strength.}{℣. Esto ei, Dómine, turris fortitúdinis.}

\elcol{℟. From the face of the enemy.}{℟. A fácie inimíci.}

\elcol{℣. Let the enemy have no advantage over \textit{him}.}{℣. Nihil profíciat inimícus in \textit{eo}.}

\elcol{℟. Nor the wicked approach to hurt \textit{him}.}{℟. Et fílius iniquitátis non appónat nocére ei.}

\elcol{℣. O Lord, hear my prayer.}{℣. Dómine, exáudi oratiónem meam.}

\elcol{℟. And let my cry come unto thee.}{℟. Et clamor meus ad te véniat.}

\elcol{℣. The Lord be with you.}{℣. Dóminus vobíscum.}

\elcol{℟. And with thy spirit.}{℟. Et cum spíritu tuo.}

\letuspray
\lett{O}{Lord} God, who through thine Apostle James hast said: Is any sick among you? Let him call for the elders of the Church and let them pray over him, anointing him with oil in the Name of the Lord: and the prayer of faith shall save the sick, and the Lord shall raise him up: and if he have committed sins, they shall be forgiven him; cure, we beseech thee, O our Redeemer, by the grace of the Holy Ghost, the weakness of this sick person; heal \textit{his} wounds and put away \textit{his} sins; cast out from \textit{him} all pain of mind and body and mercifully give back to \textit{him} full health, both inwardly and outwardly, that, being restored by the help of thy mercy, \textit{he} may return to \textit{his} duties as of old: Who with the Father and the same Holy Ghost livest and reignest God, world without end. \textit{Amen.}\par
\letuspray
\lett{L}{ook,} O Lord, we beseech thee, upon this thy \textit{servant} \textit{N.}, languishing in weakness of body, and comfort again the soul which thou hast created: that, being amended by thy chastisement, \textit{he} may feel \textit{himself} to be saved by thy healing. Through Christ our Lord. \textit{Amen.}\par
\letuspray
\lett{O}{Lord} holy, Father almighty, everlasting God, who in pouring the grace of thy blessing upon sick bodies dost preserve by thy manifold goodness thy handy-work: graciously assist us who call upon thy Name; deliver thy \textit{servant} from \textit{his} sickness, and give \textit{him} health; raise \textit{him} up by thy right hand; strengthen \textit{him} by thy might; protect \textit{him} by thy power, and with all the prosperity which \textit{he} desires restore \textit{him} to thy holy Church. Through Christ our Lord. \textit{Amen.}
\begin{rubric}
    At the end, the Priest may give brief and salutary exhortations to enable the sick person to die in the Lord, and to strengthen him to put to flight the temptations of evil spirits.
\end{rubric}
\begin{rubric}
    Finally, he should leave with him holy water and a Cross, that he may frequently look up on it, and according to his devotion kiss it and embrace it.
\end{rubric}
\begin{rubric}
    He should also warn the relatives and servants of the sick man to send at once for the Parish Priest, if the disease grows worse, that he may help the dying man, and commend his soul to God; but if death is at hand, the Priest, before he departs, shall duly commend the soul to God.
\end{rubric}
\begin{rubric}
    When this Sacrament is administered to several sick persons at the same time, the Priest shall present the Cross to each to be devoutly kissed, and shall say all the prayers which precede and follow the anointing once for all in the plural number; but he shall perform the anointings with their respective forms separately on each sick person.
\end{rubric}
\subby{After Holy Unction}\par\noindent
Return thanks now to your loving Saviour with your whole heart, for having favoured you with all these helps in your sickness. Reflect how many are carried off by sudden death, or otherwise die without the holy Sacraments, or any of the extraordinary graces which God has afforted you! Beg of him that this holy Unction may produce in you all the happy fruits for which it was instituted by the goodness of your Saviour, by healing your soul of all its weaknesses and spiritual maladies: by fortifying you against all the temptations of the enemy: by supporting and comforting you under all your pains and anguish: by preparing and disposing you for whatever may be the holy will of God in your regard: and, if he sees it expedient for you, by restoring you to your bodily health and strength. In the meantime, keep yourself, as much as you can in the company of your Saviour Jesus Christ: but let it be with the dispositions of a true penitent, often bewailing your sins at his feet, and calling upon him for mercy. Hide yourself in his Wounds and bathe yourself in his precious Blood. A truly penitent spirit will be your best security both in life and death. But then, let this be joined with a great confidence in the mercy of God, and in the merits of Jesus Christ who died for you. Keep your eyes fixed upon him: contemplate the infinite and eternal happiness he has prepared for you in his heavenly kingdom: relinquish from this moment all worldly concerns, and all desires of remaining any longer in this place of banishment : and frequently say with St. Paul: I desire to be dissolved and to be with Christ : resign yourself entirely into his hands: let the consideration of the holy will of God, the glory he has prepared for you, and the sufferings your Saviour endured for your sake, animate you to bear with patience all your sufferings. Offer up all your pains and troubles to him: accept them as a penance justly inflicted on you for your sins: and pray that they may be sanctified and accepted through him. Beg also the intercession of the Blessed Virgin, and of all the glorious Angels and Saints of God, that you may be helped by their prayers both in life and death.
\subby{Questions Proper to be Asked of the Sick, to Excite them to Make Acts of the Necessary Virtues}
%Changed to avoid heresy:
\noindent
Do you firmly believe all the Articles of Faith which the Holy, Catholic, Apostolic, Orthodox Church believes and teaches?\par
℟. I do believe them.\par\noindent
Do you firmly hope that God will be merciful to you: and that through the merits of Jesus Christ you will obtain from him the forgiveness of your sins, and life everlasting?\par
℟. I do.\par\noindent
Do you love God with your whole heart? and do you desire to love him as the blessed do in heaven?\par
℟. I do.\par\noindent
Are you, for the love of God, sorry from your heart, for every offence you have committed against him, and against your neighbour?\par
℟. I am.\par\noindent
Do you, for God's sake, forgive from your heart every one who has ever offended you, or been your enemy?\par
℟. I do.\par\noindent
Do you now, from your heart, ask pardon of every one whom you have offended by word or deed?\par
℟. I do.\par\noindent
Do you receive your present and future sufferings as penance justly inflicted on you by Almighty God; and will you endeavour to bear them with the patience becoming a Christian?\par
℟. I do so receive them: and will endeavour to bear them with patience and resignation to the holy will of God.\par\noindent
If it shall please God to restore you again to your bodily health and strength, will you, during the remainder of your life, carefully endeavour to avoid sin, and keep all his divine commandments?\par
℟. This is my determined resolution.\par
%\begin{rubric}
%    Short Acts of the most necessary Virtues to be suggested to the Sick, leisurely and distinctly, more especially when they are drawing near to their end, and cannot bear longer prayers.
%\end{rubric}
