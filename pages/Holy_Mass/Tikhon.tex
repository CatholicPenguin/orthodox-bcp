\fancyhead[C]{{\LARGE Holy Mass}}
\fancyhead[RO,LE]{}
\fancyhead[RE,LO]{}
%\fancyhead[RO,LE]{\textit{Liturgy of St. Tikhon}}
%\section{Liturgy of St. Tikhon}\label{tikhon}
\phantomsection
\addcontentsline{toc}{section}{Liturgy of St. Tikhon}
%\begin{rubric}
%If any be an open and notorious evil liver, or have done any wrong to his neighbours by word or deed, so that the Congregation be thereby offended; the Curate, having knowledge thereof, shall call him and advertise him, that in any wise he presume not to come to the Lord’s Table, until he have openly declared himself to have truly repented and amended his former naughty life, that the Congregation may thereby be satisfied, which before were offended; and that he have recompensed the parties, to whom he hath done wrong; or at least declare himself to be in full purpose so to do, as soon as he conveniently may.
%\end{rubric}
%\begin{rubric}
%The same order shall the Curate use with those betwixt whom he perceiveth malice and hatred to reign; not suffering them to be partakers of the Lord's Table, until he know them to be reconciled. And if one of the parties so at variance be content to forgive from the bottom of his heart all that the other hath trespassed against him, and to make amends for that he himself hath offended; and the other party will not be persuaded to a godly unity, but remain still in his frowardness and malice the Minister in that case ought to admit the penitent person to the Holy Communion, and not him that is obstinate.
%\end{rubric}
%\begin{rubric}
%Provided that every Minister so advertising or repelling any, as is specified in the two precedent paragraphs, shall he obliged forthwith to give an account of the same to the Bishop, and therein to obey his order and direction.
%\end{rubric}
%
%
%There is a dispute between the Missals as to where to place the Collect for Purity and the Summary of the Law. The St. Tikhon's Liturgy (as it exists) and the English Missal place it at the Altar, after the Ascent. While this is a fitting place for the Collect, it creates a liturgical novelty for the Summary of the Law. At the Epistle side is offered the prayers to God: the various propers and the sacrifice of the Epistle. In contrast, the Summary of the Law is not directed towards God (who needs no preaching of His own Law) but rather towards the people. Therefore, it is absurd to have the place of sacrifice to God (the Epistle side) be the place of exhortation to the People of God, especially if does not even turn towards the Congregation. There is an example where the Priest does fully turn away from God towards the Congregation: the \emph{Orate Frates}, which occurs in the middle of the Altar. Therefore, to avoid novelties and liturgical inconsistencies, I have placed the Collect and the Summary/Decalogue at the Foot of the Altar, as other Missals have so done (it is appropriate for the Collect, for that is where other collects are read, and it is supremely expedient for an exhortation of the People).
\subby{Collect for Purity}
\begin{rubric}
    The Priest, at the Foot of the Altar, saith the Collect for Purity, as followeth.
\end{rubric}
\lett{A}{lmighty} God, unto whom all hearts are open, all desires known, and from whom no secrets are hid; Cleanse the thoughts of our hearts by the inspiration of thy Holy Spirit, that we may perfectly love thee, and worthily magnify thy holy Name; through Christ our Lord. \textit{Amen.}
\begin{rubric}
Then shall the Priest, turning to the People, rehearse distinctly The Ten Commandments; and the People, still kneeling, shall, after every Commandment, ask God mercy for their transgressions for the time past, and grace to keep the law for the time to come.
\end{rubric}
\begin{rubric}
And \textsc{Note,} that in rehearsing the Ten Commandments, the Priest may omit that part of the Commandment which is inset.
\end{rubric}
\begin{rubric}
	The Decalogue may be omitted. But \textsc{Note,}, That when ever it is omitted, the Minister shall say the Summary of the Law, beginning, \emph{Hear what our Lord Jesus Christ saith.}
\end{rubric}
%MANUAL ADJUSTMENT:
\vspace{-1ex}
\subby{Ten Commandments}
\begin{center}
	\textsc{God spake these words, and said:}
\end{center}
%\begin{multicols}{2}
\par\noindent
    I am the \textsc{Lord} thy God; Thou shalt have none other gods but me.\par
    \textit{Lord, have mercy upon us, and incline our hearts to keep this law.}
    \par\noindent
    Thou shalt not make to thyself any graven image, nor the likeness of any thing that is in heaven above, or in the earth beneath, or in the water under the earth; thou shalt not bow down to them, nor worship them:
    \par\noindent
    \leftskip=2em
	{\small{for I the Lord thy God am a jealous God, and visit the sins of the fathers upon the children, unto the third and fourth generation of them that hate me; and show mercy unto thousands in them that love me and keep my commandments.}}
	\par
	\leftskip=0cm
    \textit{Lord, have mercy upon us, and incline our hearts to keep this law.}
    \par\noindent
    Thou shalt not take the Name of the Lord thy God in vain;
    \par\noindent
    \leftskip=2em
	{\small{for the Lord will not hold him guiltless, that taketh his Name in vain.}}
	\par
	\leftskip=0em
    \textit{Lord, have mercy upon us, and incline our hearts to keep this law.}
    \par\noindent
    Remember that thou keep holy the Sabbath-day.
    \par\noindent
    \leftskip=2em
	{\small{Six days shalt thou labour, and do all that thou hast to do; but the seventh day is the Sabbath of the Lord thy God. In it thou shalt do no manner of work; thou, and thy son, and thy daughter, thy man-servant, and thy maid-servant, thy cattle, and the stranger that is within thy gates. For in six days the Lord made heaven and earth, the sea, and all that in them is, and rested the seventh day: wherefore the Lord blessed the seventh day, and hallowed it.}}
	\par
	\leftskip=0em
    \textit{Lord, have mercy upon us, and incline our hearts to keep this law.}
    \par\noindent
    Honour thy father and thy mother;
    \par\noindent
    \leftskip=2em
	{\small{that thy days may be long in the land which the Lord thy God giveth thee.}}
	\par
	\leftskip=0em
    \textit{Lord, have mercy upon us, and incline our hearts to keep this law.}
    \par\noindent
    Thou shalt do no murder.\par
    \textit{Lord, have mercy upon us, and incline our hearts to keep this law.}
    \par\noindent
    Thou shalt not commit adultery.\par
    \textit{Lord, have mercy upon us, and incline our hearts to keep this law.}
\par\noindent
    Thou shalt not steal.\par
    \textit{Lord, have mercy upon us, and incline our hearts to keep this law.}
\par\noindent
    Thou shalt not bear false witness against thy neighbour.\par
    \textit{Lord, have mercy upon us, and incline our hearts to keep this law.}
    \par\noindent
    Thou shalt not covet.
    \par\noindent
    \leftskip=2em
	{\small{thy neighbour's house, thou shalt not covet thy neighbour's wife, nor his servant, nor his maid, nor his ox, nor his ass, nor any thing that is his.}}
	\par
	\leftskip=0em
	\textit{Lord, have mercy upon us, and write all these thy laws in our hearts, we beseech thee.}
	
%	\end{multicols}
\begin{rubric}
	Then may the Priest say the Summary of the Law, as followeth.
\end{rubric}
\subby{Summary of the Law}
\begin{center}
	{\textsc{Hear what our Lord Jesus Christ saith.}}
\end{center}

\lett{T}{hou} shalt love the Lord thy God with all thy heart, and with all thy soul, and with all thy mind. This is the first and great commandment. And the second is like unto it; Thou shalt love thy neighbour as thyself. On these two commandments hang all the Law and the Prophets.
\begin{rubric}
    Then shall the Priest turn towards the Altar and ascend, saying secretly the \emph{Take away}.
\end{rubric}
\begin{rubric}
    Then, with hands joined upon the Altar, the Priest saith, bowing (kissing the Altar at the middle), the \emph{We pray thee}.
\end{rubric}
\begin{rubric}
    At a solemn Mass, the Celebrant, before reading the Introit, shall bless the incense with the \emph{Be thou blessed}.
\end{rubric}
\begin{rubric}
	Receiving the thurible from the Deacon, he censeth the Altar, saying nothing. Then the Deacon taketh the thurible from the Celebrant and censeth him only.
\end{rubric}
\begin{rubric}
%Added clarity that the Priest should be in the middle of the Altar for the Kyrie.
    Then shall the Priest move to the Epistle side and, signing himself with the sign of the Cross, begin the Introit: which finished, at the middle of the Altar, with joined hands, he saith alternately with the Ministers the \emph{Kyrie},
\end{rubric}
\subby{Kyrie}
\elcoldent{℣. Lord, have mercy upon us.}{℣. Kyrie, eléison.}
\elcoldent{℟. Lord, have mercy upon us.}{℟. Kyrie, eléison.}
\elcoldent{℣. Lord, have mercy upon us.}{℣. Kyrie, eléison.}
\elcoldent{℟. Christ, have mercy upon us.}{℟. Christe, eléison.}
\elcoldent{℣. Christ, have mercy upon us.}{℣. Christe, eléison.}
\elcoldent{℟. Christ, have mercy upon us.}{℟. Christe, eléison.}
\elcoldent{℣. Lord, have mercy upon us.}{℣. Kyrie, eléison.}
\elcoldent{℟. Lord, have mercy upon us.}{℟. Kyrie, eléison.}
\elcoldent{℣. Lord, have mercy upon us.}{℣. Kyrie, eléison.}
\subby{Gloria in Excelsis}
\begin{rubric}
    Then the Priest shall, at the middle of the Altar, extend and join his hands and---bowing his head slightly---say, if it is to be said,
\end{rubric}
\elcol{\lett{G}{lory} be to God on high, and on earth peace, good will towards men. We praise thee, we bless thee, we worship thee,\margcolrub{Bow head.} we glorify thee, we give thanks\margcolrub{Bow head.} to thee for thy great glory, O Lord God, heavenly King, God the Father Almighty.\par
	O Lord, the only-begotten Son, Jesus Christ; O Lord God, Lamb of God, Son of the Father, that takest away the sins of the world, have mercy upon us. Thou that takest away the sins of the world, receive our prayer.\margcolrub{Bow head.} Thou that sittest at the right hand of God the Father, have mercy upon us.\par
	For thou only art holy; thou only art the Lord; thou only, O Christ, with the Holy Ghost, {\ding{64}} art most high in the glory of God the Father. Amen.}
    {\lett{G}{l\smash{ó}ria} in excélsis Deo. Et in terra pax homínibus bon{\ae} voluntátis. Laudámus te. Benedícimus te. Adorámus te.\marglatincolrub{Bow head.} Glorificámus te. Grátias ágimus tibi\marglatincolrub{Bow head.} propter magnam glóriam tuam. Dómine Deus, Rex c{\ae}léstis, Deus Pater omnípotens.\par
    Dómine Fili unigénite, Jesu Christe. Dómine Deus, Agnus Dei, Fílius Patris. Qui tollis peccáta mundi, miserére nobis. Qui tollis peccáta mundi, súscipe deprecatiónem nostram.\marglatincolrub{Bow head.} Qui sedes ad déxteram Patris, miserére nobis.\par
    Quóniam tu solus Sanctus. Tu solus Dóminus. Tu solus Altíssimus, Jesu Christe. Cum Sancto Spíritu {\ding{64}} in glória Dei Patris. Amen.}
%MANUAL ADJUSTMENT:
\clearpage
\begin{rubric}
    Then shall the Priest kiss the Altar in the middle, turn to the People, extend his hands, and say,
\end{rubric}
\elcol{℣. The Lord be with you.}{℣. Dóminus vobíscum.}
\elcol{℟. And with thy spirit.}{℟. Et cum spíritu tuo.}
\elcol{℣. Let us pray.}{℣. Orémus.}
\subby{Collect}
\begin{rubric}
	Then shall the Priest move to the Epistle corner of the Altar and pray the Collect(s) of the Day.
\end{rubric}
\subby{Epistle}
\begin{rubric}
	The Epistle for the Day shall then be read by the Subdeacon, first saying, \emph{The Epistle is written in the \emph{--} Chapter of \emph{--}, beginning at the \emph{--} Verse.}
\end{rubric}
\begin{rubric}
	The Epistle ended, he shall say, \emph{Here endeth the Epistle}, the people responding, \emph{Thanks be to God}.
\end{rubric}
\begin{rubric}
    The Gradual and Alleluia (or Tract) and (if provided) Sequence is here chanted by the Choir.
\end{rubric}
\subby{Gospel}
\begin{multicols}{2}
\begin{rubric}
    These being ended, if it be a Solemn Mass, the Deacon shall then place the book of the Gospels on the middle of the Altar, and the Celebrant shall bless the incense with the \emph{Be thou blessed}, as above.
\end{rubric}
\begin{rubric}
	Then shall the Deacon, kneeling before the Altar, say with joined hands the \emph{Cleanse my heart} continuing to the \emph{Bid, sir}, with the priest saying \emph{The Lord be in thy}.
\end{rubric}
\begin{rubric}
    Having received the blessing, the Deacon kisseth the hand of the Celebrant. And going with the other Ministers, with the incense and the lights, to the place of the Gospel, he standeth with joined hands, saying what is below.
\end{rubric}
\begin{rubric}
    If, however, the Priest celebrate without Deacon and Subdeacon, when the book has been carried to the other corner of the Altar, he boweth in the middle, and with joined hands, saith the \emph{Cleanse my heart} and the \emph{Bid, Lord} and \emph{The Lord be in my}.
\end{rubric}
\end{multicols}
\begin{rubric}
	Then, all the People standing, the Priest, turning to the Book, shall say with joined hands,
\end{rubric}
\elcol{℣. The Lord be with you.}{℣. Dóminus vobíscum.}
\elcol{℟. And with thy spirit.}{℟. Et cum spíritu tuo.}
\elcol{℣. The Beginning (or, Continuation) {\ding{66}} of the Holy Gospel according to \textit{N.}}{℣. Inítium (vel, Sequéntia) {\ding{66}} sancti Evangélii secúndum \textit{N.}}
\elcol{℟. Glory be to thee, O Lord.}{℟. Glória tibi, Dómine.}
\begin{rubric}
    Then shall the Priest sign the book with the thumb of his right hand at the beginning of the Gospel text which he is to read, then himself on the forehead, the mouth, and the breast: and while the Ministers respond, he censeth the book thrice, then readeth the Gospel with joined hands.
\end{rubric}
\begin{rubric}
    At the end of the Gospel, the Ministers respond,
\end{rubric}
\elcol{℟. Praise be to thee, O Christ.}{℟. Laus tibi, Christe.}
\begin{rubric}
    Then shall the Subdeacon carry the book to the Priest, who kisseth the Gospel text, saying: \emph{Through the words of the Gospel may our sins be blotted out.} Then the Priest is censed by the Deacon.\par
    \textsc{Note,} In Masses of the Dead, \emph{Cleanse} is said, but a blessing is not asked, lights are not carried, and the Celebrant doth not kiss the book.
\end{rubric}
\begin{rubric}
    Then, in the middle of the Altar, extending, raising, and joining his hands, the Priest shall say, if it is to be said, \emph{I believe in one God}, proceeding with joined hands.
\end{rubric}
\subby{Nicene Creed}\label{NiceneCreed}
\elcol{\lett{I}{believe} in one God\margcolrub{Bow head to Cross.} the Father Almighty, Maker of heaven and earth, And of all things visible and invisible:\par
    And in one Lord Jesus Christ,\margcolrub{Bow head to Cross.} the only-begotten Son of God; Begotten of his Father before all worlds, God of God, Light of Light, Very God of very God; Begotten, not made; Being of one substance with the Father; By whom all things were made: Who for us men and for our salvation came down from heaven, \inrub{Everyone genuflects.} And was incarnate by the Holy Ghost of the Virgin Mary, And was made man: \inrub{Everyone rises.} And was crucified also for us under Pontius Pilate; He suffered and was buried: And the third day he rose again according to the Scriptures: And ascended into heaven, And sitteth on the right hand of the Father: And he shall come again, with glory, to judge both the quick and the dead; Whose kingdom shall have no end.\par
    And I believe in the Holy Ghost, The Lord, and Giver of Life, Who proceedeth from the Father; Who with the Father and the Son together is worshiped\margcolrub{Bow head to Cross.} and glorified; Who spake by the Prophets: And I believe one, holy, catholic, and apostolic Church: I acknowledge one Baptism for the remission of sins: And I look for the Resurrection of the dead: {\ding{64}} And the Life of the world to come. Amen.}
    {\lett{C}{redo} in unum Deum,\marglatincolrub{Bow head to Cross.} Patrem omnipoténtem, factórem c{\ae}li et terr{\ae}, visibílium ómnium et invisibílium.\par
    Et in unum Dóminum Jesum Christum,\marglatincolrub{Bow head to Cross.} Fílium Dei unigénitum. Et ex Patre natum ante ómnia sǽcula. Deum de Deo, lumen de lúmine, Deum verum de Deo vero. Génitum, non factum, consubstantiálem Patri: per quem ómnia facta sunt. Qui propter nos hómines et propter nostram salútem descéndit de c{\ae}lis. \inrub{Everyone genuflects.} Et incarnátus est de Spíritu Sancto ex María Vírgine: Et homo factus est. \inrub{Everyone rises.} Crucifíxus étiam pro nobis: sub Póntio Piláto passus, et sepúltus est. Et resurréxit tértia die, secúndum Scriptúras. Et ascéndit in c{\ae}lum: sedet ad déxteram Patris. Et íterum ventúrus est cum glória judicáre vivos et mórtuos: cujus regni non erit finis.\par
    Et in Spíritum Sanctum, Dóminum et vivificántem: qui ex Patre procédit. Qui cum Patre et Fílio simul adorátur\marglatincolrub{Bow head to Cross.} et conglorificátur: qui locútus est per Prophétas. Et unam sanctam cathólicam et apostólicam Ecclésiam. Confíteor unum baptísma in remissiónem peccatórum. Et exspécto resurrectiónem mortuórum. {\ding{64}} Et vitam ventúri s{\ae}culi. Amen.}
\begin{rubric}
Then shall be declared unto the People what Holy-days, or Fasting-days, are in the week following to be observed; and (if occasion be) shall Notice be given of the Communion, and of the Banns of Matrimony, and other matters to be published.
\end{rubric}

\begin{rubric}
Here, or immediately after the Creed, may be said the Bidding Prayer, or other authorized Prayers and intercessions.
\end{rubric}

\begin{rubric}
	Then followeth the Sermon.
\end{rubric}

\begin{rubric}
    Then shall the Priest ascend the Altar, kiss it, turn towards the People, and say,
\end{rubric}
\elcol{℣. The Lord be with you.}{℣. Dóminus vobíscum.}
\elcol{℟. And with thy spirit.}{℟. Et cum spíritu tuo.}
\elcol{℣. Let us pray.}{℣. Orémus.}
\subsec{Offertory}
\begin{multicols}{2}    
\begin{rubric}
    The Priest then saith the Offertory Verse.\par
    \textsc{Note,} At High Mass, a hymn may be sung while the Priest prepareth the Oblations in the Offertory, and while the Collection is taken if there be one.
\end{rubric}
\begin{rubric}
    Then shall the Deacon present the Paten with the Bread to the Celebrant, which he then offereth, saying: \emph{Receive, O holy Father}.
\end{rubric}
%MANUAL ADJUSTMENT:
\columnbreak
\begin{rubric} 
    Then, making a cross with the same Paten, he shall place the Bread upon the Corporal (or the Paten with the Bread upon the Corporal).
\end{rubric}
\begin{rubric}
	Then the Deacon shall minister the Wine, and the Subdeacon the water in the Cup. The Priest (blessing with the sign of the Cross the water to be mixed in the Cup, unless it be a Requiem Mass) shall say: \emph{O God, who didst wonderfully}.
\end{rubric}
\begin{rubric}
    Then shall the Priest receive the Cup and offer it, saying, \emph{We offer unto thee}.
\end{rubric}
\begin{rubric}
    Then shall he make the sign of the Cross with the Cup, place it upon the Corporal, and cover it with the Pall. Then, with hands joined upon the Altar, he saith, bowing slightly, \emph{In a humble spirit}.
\end{rubric}
\begin{rubric}
    Then shall the Priest---standing upright---extend his hands, raise them, and join them; and lifting his eyes to heaven and lowering them immediately, he saith, \emph{Come, O Sanctifier}.
\end{rubric}
\begin{rubric}
    If he be celebrating solemnly, the Priest blesseth incense, saying, \emph{Through the intercession}.
\end{rubric}
\begin{rubric}
    Receiving the Thurible from the Deacon, he censeth the Oblations, in the manner prescribed in the General Rubrics, saying: \emph{May this incense}.\par
    Then he censeth the Altar, saying: \emph{Let my prayer}.\par
    While he returneth the Thurible to the Deacon, he saith, \emph{The Lord kindle}.\par
    Then the Priest is censed by the Deacon, and afterwards the others in order.
\end{rubric}
\begin{rubric}
    Meanwhile, the Priest washeth his hands saying, \emph{I will wash}.\par
    \textsc{Note,} In Requiem Masses, and during Passiontide in Masses of the Season, \emph{Glory be} is omitted.
\end{rubric}
\begin{rubric}
    Then bowing slightly, in the middle of the Altar with hands joined upon it, he saith, \emph{Receive, O holy Trinity}.
\end{rubric}
\end{multicols}
\begin{rubric}
    Then shall the Priest kiss the Altar and, turning to the People, extend and join his hands, and say, raising his voice a little,
\end{rubric}
\elcol{℣. Pray, brethren: that my sacrifice and yours may be acceptable to God the Father almighty.}{℣. Oráte, fratres: ut meum ac vestrum sacrifícium acceptábile fiat apud Deum Patrem omnipoténtem.}
\elcol{℟. The Lord receive the sacrifice at thy hands, to the praise and glory of his name, and to our benefit, and that of all his holy Church.}{℟. Suscípiat Dóminus sacrifícium de mánibus tuis ad laudem et glóriam nominis sui, ad utilitátem quoque nostram, totiúsque Ecclési{\ae} su{\ae} sanct{\ae}.}
\begin{rubric}
    The Priest saith in a low voice: \emph{Amen.}
\end{rubric}
\begin{rubric}
	Then shall the Priest, with hands extended, immediately (without \emph{Let us pray}), add the Secret Prayers.\par
	\textsc{Note,} When these are ended, he saith in a loud voice,
\end{rubric}
℣. Throughout all ages, world without end.\par
℟. Amen.
\begin{rubric}
    Then shall the Deacon say,
\end{rubric}
\centerline{Let us pray for the whole state of Christ's Church.}
\begin{rubric}
    The Priest then saith, with hands extended, the following.
\end{rubric}
\lett{A}{lmighty} and everliving God, who by thy holy Apostle hast taught us to make prayers, and supplications, and to give thanks for all men; We humbly beseech thee most mercifully to accept our (alms and) oblations, and to receive these our prayers, which we offer unto thy Divine Majesty; beseeching thee to inspire continually the Universal Church with the spirit of truth, unity, and concord: And grant that all those who do confess thy holy Name may agree in the truth of thy holy Word, and live in unity and godly love.
\needspace{4\baselineskip}
\lett{W}{e} beseech thee also, so to direct and dispose the hearts of all Christian Rulers, that they may truly and impartially administer justice, to the punishment of wickedness and vice, and to the maintenance of thy true religion, and virtue.
\needspace{4\baselineskip}
%Mention of bishops imported from 1929 Scottish BCP:
\lett{G}{ive} grace, O heavenly Father, to all Bishops and other Ministers, and especially to thy servant \emph{N.} our Metropolitan (and \emph{N.} our Bishop), that \textit{they} may, both by \textit{their} life and doctrine, set forth thy true and lively Word, and rightly and duly administer thy holy Sacraments.
\needspace{4\baselineskip}
\lett{A}{nd} to all thy People give thy heavenly grace; and especially to this congregation here present; that, with meek heart and due reverence, they may hear, and receive thy holy Word; truly serving thee in holiness and righteousness all the days of their life.
\needspace{4\baselineskip}
\lett{A}{nd} we most humbly beseech thee, of thy goodness, O Lord, to comfort and succour all those who, in this transitory life, are in trouble, sorrow, need, sickness, or any other adversity.
\needspace{4\baselineskip}
%Mention of the Saints imported from 1929 Scottish BCP:
\lett{A}{nd} we also bless thy holy Name for all thy servants departed this life in thy faith and fear; beseeching thee to grant them continual growth in thy love and service, and to give us grace so to follow their good examples, chiefly the Blessed Virgin Mary, Mother of thy Son Jesus Christ our Lord and God, and the Holy Patriarchs, Prophets, Apostles, and Martyrs, that with them we may be partakers of thy heavenly kingdom.
\begin{rubric}
	 The Priest placeth both hands upon the Altar, outside the Corporal, and continueth,
\end{rubric}
\needspace{4\baselineskip}
\lett{G}{rant} this, O Father, for Jesus Christ's sake, our only Mediator and Advocate.\par
\secondline{℟. Amen.}
%This is a difficult situation to discern. The Prayer Book Traditions reflects the uncertainty as to where to place the Confession. The current American placement is reflective of Cranmer's Protestant theology and liturgical sense, since the origins come from placing the Confession close to the reception of Communion (which was immediately after the Words of Institution). The 1928 reflects a more catholic understanding of the liturgy, including the epiklesis and the other integral parts of the Eucharistic Sacrifice before the Reception of Holy Communion. This understanding was more present in the 1549 which placed the Confession in a congruous place as the Roman tradition: after the Sacrifice and before the Communion of the Faithful. While this is obscured here due to the Byzantine devotion imposed, this is only a temporary imposition and the overall shape of the liturgy (due to the Orthodox theology present here) is better preserved in its Catholic and even Prayer Book form by placing the Confession immediately before the `Ecce Agnus Dei', which is where the Second Confiteor would go in the Roman Rite.
%MANUAL ADJUSTMENT:
\clearpage
\subsec{Sursum Corda}
\begin{rubric}
    Then shall the Priest begin the Preface, with both hands placed apart on the Altar.
\end{rubric}
\elcol{℣. The Lord be with you.}{℣. Dóminus vobíscum.}
\elcol{℟. And with thy spirit.}{℟. Et cum spíritu tuo.}
\elcol{℣. Lift up your hearts.\margcolrub{He raises his hands a little.}}{℣. Sursum corda.\marglatincolrub{He raises his hands a little.}}
\elcol{℟. We lift them up unto the Lord.}{℟. Habémus ad Dóminum.}
\elcol{℣. Let us give thanks unto our Lord God.\margcolrub{He joins his hands before his breast, \& bows his head.}}{℣. Grátias agámus Dómino, Deo nostro.\marglatincolrub{He joins his hands before his breast, \& bows his head.}}
\elcol{℟. It is meet and right so to do.\margcolrub{He separates his hands.}}{℟. Dignum et justum est.\marglatincolrub{He separates his hands.}}
\subby{Preface}
\lett{I}{t} is very meet, right, and our bounden duty, that we should at all times, and in all places, give thanks unto thee, O Lord, Holy Father, Almighty, Everlasting God.
\begin{rubric}
    The proper preface is then here said (p. \pageref{prefaces})---if there be one---concluding with the following.
\end{rubric}
\label{PrefaceEnding}
\lett{T}{herefore} with Angels and Archangels, and with all the company of heaven, we laud and magnify thy glorious Name; evermore praising thee, and saying:\margrub{He joins his hands \& bows.}%\textsuperscript{\alph{margcount}}

%\hspace{0.98\textwidth}
%\rlap{\parbox{\marginparwidth}{\raggedright \itshape\scriptsize\textsuperscript{\alph{margcount}}%
%He joins his hands \& bows.}} % Use \parbox for multiline text

%\stepcounter{margcount}
%\stepcounter{latinrubric}

\subby{Sanctus}
\elcol{\lett{H}{oly, Holy, Holy,} Lord God of hosts, Heaven and earth are full of thy glory: Glory be to thee, O Lord Most High. Blessed\margcolrub{He stands upright \& signs himself.} {\ding{64}} is he that cometh in the Name of the Lord. Hosanna in the Highest.}
{\lett{S}{anctus, Sanctus, Sanctus,} Dóminus, Deus Sábaoth. Pleni sunt c{\ae}li et terra glória tua. Hosánna in excélsis. Benedíctus,\marglatincolrub{He stands upright \& signs himself.} {\ding{64}} qui venit in nómine Dómini. Hosánna in excélsis.}
\clearpage
\checkoddpage
\ifoddpage \thispagestyle{empty}
~\clearpage\fi
\thispagestyle{empty}
\includegraphics[width=98mm, height=135mm]{Canon.eps}
\setcounter{margcount}{1}
\clearpage
%\reversemarginpar
\subby{Canon of the Mass}
\begin{rubric}
    The Priest, extending, raising somewhat, and joining his hands, raising his eyes towards heaven, and immediately lowering them, shall begin the Canon, as followeth.
\end{rubric}
\lett{A}{ll} glory be to thee, Almighty God, our heavenly Father, for that thou, of thy tender mercy, didst give thine only Son Jesus Christ to suffer death upon the Cross for our redemption; who (by his own oblation of himself once offered) made a full, perfect, and sufficient sacrifice, oblation, and satisfaction, for the sins of the whole world; and did institute, and in his holy Gospel command us to continue, a perpetual memory of that his precious death and sacrifice, until his coming again:
\subbysub{Words of Institution}\noindent
\lett{F}{or} in the night in which he was betrayed,\margrub{He takes \& holds the Bread.} he took Bread;\margrub{He looks up to heaven \& bows his head.} and when he had given {\ding{64}} thanks, he brake it, and gave it to his disciples, saying, Take, eat,\margrub{He holds the Bread between both of his thumbs \& forefingers.}
\begin{center}
\large{This is my Body, which is given for you; Do this in remembrance of me.}
\end{center}
\begin{rubric}
    Then shall the Priest immediately genuflect, briefly elevate the Bread for the People to see, replace it upon the corporal (or Paten), genuflect again, and then immediately continue,\par
    \textsc{Note,} From henceforth, the Priest doth not disjoin his forefingers \& thumbs until the ablutions.
\end{rubric}\par\noindent
Likewise, after supper,\margrub{The Priest uncovers the Cup, holds it with both hands, \& bows his head.} he took the Cup;\margrub{He holds the Cup in his left hand.} and when he had given {\ding{64}} thanks, he gave it to them, saying, Drink ye all of this; for\margrub{He holds the Cup slightly raised.}
\begin{center}
\large{This is my Blood of the New Testament, which is shed for you, and for many, for the remission of sins;}
\end{center}
\begin{rubric}
    Then shall the Priest place the Cup back upon the corporal and immediately say,
\end{rubric}\par\noindent
    \centerline{Do this, as oft as ye shall drink it, in remembrance of me.}
\begin{rubric}
    Then shall the Priest immediately genuflect, briefly elevate the Cup, replace the Cup upon the corporal, cover it, genuflect again, and continue with hands extended.
\end{rubric}
\subbysub{Recollection}
%Added signs of the Cross from American \& Anglican Missal:
\lett{W}{herefore} O Lord and heavenly Father, according to the institution of thy dearly beloved Son our Saviour Jesus Christ, we, thy humble servants, do celebrate and make here before thy Divine Majesty, with these thy holy {\ding{64}} gifts, which we now offer unto thee, the memorial thy Son hath commanded us to make; having in remembrance his blessed passion and precious death, his mighty resurrection and glorious ascension; rendering unto thee most hearty thanks for the innumerable benefits procured unto us by the same.
\subbysub{Invocation}
\begin{rubric}
	Then shall the Priest uncover the Cup, bow profoundly, join his hands upon the Altar, and continue,
\end{rubric}
%Epiklesis of St. James through the 1929 Scottish BCP:
%Crosses supplied from the Anglican Missal.
%Moving the Pall from precedent in the LAP Missal:
\lett{A}{nd} we thine unworthy servants beseech thee, most merciful Father, to hear us,\margrub{He stands upright and imposes his hands over the Gifts.} and to send thy Holy Spirit upon us and upon the {\ding{64}} se thy gifts and creatures of Br {\ding{64}} ead and W {\ding{64}} ine, that, being blessed and hallowed by his life-giving power, they may become the Bo {\ding{64}} dy and Bl {\ding{64}} ood of thy most dearly beloved Son, to the end that all who shall receive the same may be sanctified both in body and soul, and preserved unto everlasting life.
\begin{rubric}
	Then shall the Priest cover the Cup, genuflect, and continue, with hands extended,
\end{rubric}
%From Scottish 16-something.
%\lett{H}{ear} us, O merciful Father, we most humbly beseech thee, and of thy almighty goodness, vouchsafe\margrub{He imposes his hands over the Bread \& Wine, right hand over left.} so to bl {\ding{64}} ess and sanct {\ding{64}} ify, with thy Word and Holy Spirit, these thy gifts and creatures of Bread and Wine, that they may %be unto us
%become the Bo {\ding{64}} dy and Bl {\ding{64}} ood of thy most dearly beloved Son; so that we receiving them according to thy Son our Saviour Jesus Christ's holy institution, in remembrance of his death and passion, may be partakers of the same his most precious Body and Blood.
%Interpolation of the Liturgy of St. Tikhon \& 1928 BCP.
%Grant that we, receiving them according to thy Son our Saviour Jesus Christ's holy institution, in remembrance of his Death and Passion, may be partakers of his most blessed Body and Blood.

%\lett{A}{nd} we most humbly beseech thee, O merciful Father, to hear us; and, of thy almighty goodness, vouchsafe to bless and sanctify, with thy Word and Holy Spirit, these thy gifts and creatures of bread and wine; that we, receiving them according to thy Son our Saviour Jesus Christ’s holy institution, in remembrance of his death and passion, may be partakers of his most blessed Body and Blood.

%\lett{A}{nd} we most humbly beseech thee, O merciful Father, to hear us: and, of thy almighty goodness, vouchsafe to send down thy Holy Spirit upon these thy gifts and creatures of Bread and Wine, that they may be changed into the Bo {\ding{64}} dy and Bl {\ding{64}} ood of thy most dearly beloved Son.\margrub{He extends his hands.} Grant that we, receiving them according to thy Son our Saviour Jesus Christ's holy institution, in remembrance of his Death and Passion, may be partakers of his most blessed Body and Blood.
\par
\needspace{4\baselineskip}
\lett{A}{nd} we earnestly desire thy fatherly goodness, mercifully to accept this our Sacrifice of Praise and Thanksgiving; most humbly beseeching thee to grant that, by the merits and death of thy Son Jesus Christ, and through faith in his Blood, we, and all thy whole Church, may obtain remission of our sins, and all other benefits of his Passion.
\subbysub{Commemoration for the Departed}
\lett{R}{emember} also, O Lord, thy servants and handmaids \emph{N.} and \emph{N.}, who have gone before us with the sign of faith, and rest in the sleep of peace.
\begin{rubric}
    Then shall the Priest join his hands, pray for the dead he hath in mind, then extend his hands and say,
\end{rubric}\par\noindent
To them, O Lord, and to all that rest in Christ, we beseech thee to grant a place of refreshing, light, and peace.
%The English & American Missals lacks this addition, being only in the Anglican Missal tradition:
%\lett{A}{nd} vouchsafe to grant some part and fellowship with thy holy Apostles and Martyrs: with John, Stephen, Matthias, Barnabas, Ignatius, Alexander, Marcellinus, Peter, Felicitas, Perpetua, Agatha, Lucy, Agnes, Cecilia, Anastasia, and with all thy Saints: within whose fellowship we beseech thee admit us.
%\lett{R}{emember} Lord, also the souls of thy servants and handmaidens, which are gone before us with the mark of faith, and rest in the sleep of peace. \inrub{The Priest now prays for the souls of the dead.} We beseech thee, O Lord, that unto them, and unto all such as rest in Christ, thou wilt grant a place of refreshing, of light, and of peace. And vouchsafe to give unto us some portion and fellowship with thy holy Apostles and Martyrs; with John, Stephen, Matthias, Barnabas, Ignatius, Alexander, Marcellinus, Peter, Felicitas, Perpetua, Agatha, Lucia, Agnes, Cecilia, Anastasia, and with all thy Saints; within whose fellowship we beseech thee to admit us.
\subbysub{Oblation}
\lett{A}{nd} here we offer and present unto thee, O Lord, our selves, our souls and bodies, to be a reasonable, holy, and living sacrifice unto thee; humbly beseeching thee, that we, and all others who shall be partakers of this Holy Communion, may worthily receive the most precious Bo {\ding{64}} dy and Bl {\ding{64}} ood\margrub{After signing the Gifts, he signs himself.} of thy Son Jesus Christ, be filled with thy grace and heavenly benediction, and made one body with him, that he may dwell in us, and we in him.
%\subbysub{Commemorations}
%Liturgy of St. Tikhon:
%\lett{B}{e} mindful also, O Lord, of thy servants who are gone before us with the sign of faith, and who rest in the sleep of peace, \textit{N.} and \textit{N.} To them O Lord, and to all who rest in Christ grant we pray thee a place of refreshment, light, and peace. To us sinners also, thy servants, confiding in the multitude of thy mercies, grant some lot and partnership with thy holy Apostles and Martyrs: John, Stephen, Matthias, Barnabas, Ignatius, Alexander, Marcellinus, Peter, Felicitas, Perpetua, Agatha, Lucia, Agnes, Cecilia, Anastasia, and with all thy Saints, into whose company we pray thee of thy mercy to admit us.
\par
\needspace{4\baselineskip}
\lett{A}{nd} although we are unworthy, through our manifold sins, to offer unto thee any sacrifice; yet we beseech thee to accept this our bounden duty and service; not weighing our merits, but pardoning our offences.\margrub{He joins his hands \& bows his head profoundly.}
\par
\needspace{4\baselineskip}
\lett{T}{hrough} Jesus Christ our Lord;\margrub{He genuflects} by {\ding{64}} whom, and with wh {\ding{64}} om, in the unity of the Holy {\ding{64}} Ghost, all ho {\ding{64}} nour and gl {\ding{64}} ory be unto thee,\margrub{He uncovers the Cup, elevates the Host \& Cup to the height of his breast then replacing them on the Altar, covers the Cup, genuflects.} O Father Almighty, world without end.\margrub{He joins his hands.}\par
℟. Amen.
%MANUAL ADJUSTMENT:
\vspace{-1ex}
\subby{Lord's Prayer}\par\noindent
Let us pray. And now, as our Saviour Christ hath taught us, we are bold to say,\margrub{He extends his hands.}
\lett{O}{ur} Father, who art in heaven, Hallowed be thy Name. Thy kingdom come. Thy will be done, on earth, As it is in heaven. Give us this day our daily bread. And forgive us our trespasses, As we forgive those who trespass against us. And lead us not into temptation, But deliver us from evil. For thine is the kingdom, and the power, and the glory, for ever and ever. Amen.
\begin{rubric}
    Unless the Priest consecrated on the Paten, he takes the Paten between the fore and middle fingers of his right hand, and holds it upright upon the Altar.
\end{rubric}
\needspace{4\baselineskip}
\lett{D}{eliver} us, O Lord, we beseech thee, from all evils, past, present, and to come: and at the intercession of the blessed and glorious ever Virgin Mary, Mother of God, with thy blessed Apostles Peter and Paul, and with Andrew, and all the Saints,\margrub{He signs himself (with the Paten, if he consecrated on the Corporal).} favourably grant peace in our days: that by the help of thine availing mercy we may ever both be free from sin and safe from all distress.
\begin{rubric}
    If the Host be not already on the Paten, the Priest is to slide the Paten underneath the Host. He then shall uncover the Cup, genuflect, and break the Host in half over the Cup, saying,
\end{rubric}\par\noindent
Through the same Jesus Christ, thy Son our Lord.
\begin{rubric}
    The Priest placeth the half in his right hand upon the Paten. He then breaketh a particle off from the half in his left hand, saying,
\end{rubric}\par\noindent
Who liveth and reigneth with thee in the unity of the Holy Ghost, God.
\begin{rubric}
    Then shall the Priest join the other half, which he holdeth in his left hand, to the half laid upon the Paten, and retaining the small particle in his right hand over the Cup, which he holdeth with his left by the knop below the cup, say in an audible voice, or sing,
\end{rubric}\par\noindent
Throughout all ages world without end.\par
℟. Amen.\setcounter{margcount}{1}\setcounter{latinrubric}{1}
\elcol{℣. The peace\margcolrub{With the same particle, he signs thrice over the Cup.} {\ding{64}} of the Lord be {\ding{64}} alway with {\ding{64}} you.}{℣. Pax\marglatincolrub{With the same particle, he signs thrice over the Cup.} {\ding{64}} Dómini sit {\ding{64}} semper vobís {\ding{64}} cum.}
\elcol{℟. And with thy spirit.}{℟. Et cum spíritu tuo.}
\begin{rubric}
    The Priest, putting the same particle into the Cup, saith, in the secret voice:
\end{rubric}
\lett{M}{ay} this commixture and consecration of the Body and Blood of our Lord Jesus Christ be to us who receive it unto everlasting life. Amen.
\subby{Agnus Dei}
\begin{rubric}
    Then shall the Priest cover the Cup, genuflect, and---bowing to the Sacrament---join his hands and strike his breast thrice, saying in an audible voice:
\end{rubric}
\elcol{\lett{O}{Lamb} of God, that takest away the sins of the world, have mercy upon us.}{\lett{A}{gnus} Dei, qui tollis peccáta mundi: miserére nobis.}
\elcol{O Lamb of God, that takest away the sins of the world, have mercy upon us.}{Agnus Dei, qui tollis peccáta mundi: miserére nobis.}
\elcol{O Lamb of God, that takest away the sins of the world, grant us thy peace.}{Agnus Dei, qui tollis peccáta mundi: dona nobis pacem.}
\begin{rubric}
	In Requiem Masses, \emph{Have mercy upon us} is replaced by \emph{grant them rest} and \emph{grant us thy peace} by \emph{grant them rest everlasting}.
\end{rubric}
%This is the original location of the Prayer Book Confession, mirroring the placement of the Roman Confession. However, in order to emphasise Reformed theology, the later English Prayer Books placed reception of Holy Communion to immediately after the Words of Institution, requiring the Confession to be placed before the Canon. Even though the American tradition developed a more Catholic Eucharistic rite, the Reformed placement of the Confession remained. Here, this is corrected in order to better express our Orthodox Catholic theology.
\subsec{Confession}
\begin{rubric}
    Then the Deacon turneth to the People and saith to those who come to receive the Holy Communion,
\end{rubric}
\lett{Y}{e} who do truly and earnestly repent you of your sins, and are in love and charity with your neighbours, and intend to lead a new life, following the commandments of God, and walking from henceforth in his holy ways; Draw near with faith, and take this holy Sacrament to your comfort; and make your humble confession to Almighty God, devoutly kneeling.
\begin{rubric}
Then shall this General Confession be made, by the Priest, bowing, and all those who are minded to receive the Holy Communion, humbly kneeling.
\end{rubric}
\needspace{4\baselineskip}
\lett{A}{lmighty} God, Father of our Lord Jesus Christ, Maker of all things, Judge of all men; We acknowledge and bewail our manifold sins and wickedness, Which we, from time to time, most grievously have committed, By thought, word, and deed, Against thy Divine Majesty, Provoking most justly thy wrath and indignation against us. We do earnestly repent, And are heartily sorry for these our misdoings; The remembrance of them is grievous unto us; The burden of them is intolerable. Have mercy upon us, Have mercy upon us, most merciful Father; For thy Son our Lord Jesus Christ's sake, Forgive us all that is past; And grant that we may ever hereafter Serve and please thee in newness of life, To the honour and glory of thy Name; Through Jesus Christ our Lord. Amen.
\begin{rubric}
	Then shall the Priest (Bishop if he be present) stand up and, turning to the People, say,
\end{rubric}
\needspace{4\baselineskip}
\lett{A}{lmighty} God, our heavenly Father, who of his great mercy hath promised forgiveness of sins to all those who with hearty repentance and true faith turn unto him; Have mercy upon you; pardon {\ding{64}} and deliver you from all your sins; confirm and strengthen you in all goodness; and bring you to everlasting life; through Jesus Christ our Lord.

℟. Amen.
\subsec{Comfortable Words}
\begin{rubric}
Then shall the Priest, facing the People, say,
\end{rubric}\noindent
\begin{center}
	\textsc{Hear what comfortable words our Saviour Christ saith unto all who truly turn to him.}
\end{center}
\par\noindent
Come unto me, all ye that travail and are heavy laden, and I will refresh you. \vr{Matt. 11:28}
\par\noindent
    So God loved the world, that he gave his only-begotten Son, to the end that all that believe in him should not perish, but have everlasting life. \vr{John 3:16}
    \par\noindent
    \begin{center}
		\textsc{Hear also what Saint Paul saith.}
	\end{center}
\par\noindent
    This is a true saying, and worthy of all men to be received, That Christ Jesus came into the world to save sinners. \inrub{1 Tim. 1:15}
\par\noindent
    \begin{center}
		\textsc{Hear also what Saint John saith.}
	\end{center}
    \par\noindent
    If any man sin, we have an Advocate with the Father, Jesus Christ the righteous; and he is the Propitiation for our sins. \inrub{1 John 2:1-2}
\begin{rubric}
    Then the Priest, turning to the Altar, bowing with hands joined upon the Altar, saith in the secret voice,
\end{rubric}
    \lett{O}{Lord} Jesu Christ, who saidst to thine Apostles: Peace I leave with you, my peace I give unto you: regard not my sins but the faith of thy Church; and vouchsafe to grant her peace and unity according to thy will: Who livest and reignest God, throughout all ages, world without end. Amen.
\begin{rubric}
    If the \emph{Pax} be given, the Priest is to kiss the Altar and---giving the \emph{Pax}---say: \emph{Peace be with thee.} with the response \emph{And with thy spirit.}
\end{rubric}
\subby{Prayer of Humble Access}
\begin{rubric}
    The Priest alone then saith,
\end{rubric}
\lett{W}{e} do not presume to come to this thy Table, O merciful Lord, trusting in our own righteousness, but in thy manifold and great mercies. We are not worthy so much as to gather up the crumbs under thy Table. But thou art the same Lord, whose property is always to have mercy: Grant us therefore, gracious Lord, so to eat the flesh of thy dear Son Jesus Christ, and to drink his blood, that our sinful bodies may be made clean by his Body, and our souls washed through his most precious Blood, and that we may evermore dwell in him, and he in us.\par
℟. Amen.
\subby{Communion of the Priest}
\begin{multicols}{2}
\begin{rubric}
    Then the Priest shall say the \emph{O Lord Jesu Christ} and \emph{Let the partaking}.
\end{rubric}
\begin{rubric}
    He then genuflecteth and saith: \emph{I will receive the bread of heaven, and call upon the name of the Lord}.
\end{rubric}
\begin{rubric}
    Then, bowing slightly, the Priest shall take both parts of the Host between the thumb and forefinger of his left hand, and place the Paten between the same forefinger and middle finger, and striking his breast three times with his right hand, say thrice, devoutly and humbly, raising his voice slightly:
\end{rubric}\par\noindent
    Lord, I am not worthy, \inrub{Proceeding in the secret voice:} that thou shouldest come under my roof: but speak the word only, and my soul shall be healed.
\begin{rubric}
After signing himself with his right hand with the Host over the Paten, he saith,
\end{rubric}\par\noindent
The Body of our Lord Jesus Christ preserve my soul unto everlasting life. Amen.
\begin{rubric}
And bowing, he reverently taketh both parts of the Host. After consuming the Host, he is to place the Paten down upon the Corporal, and raising himself, join his hands, and remain still for a short time in meditation on the Most Holy Sacrament.
\end{rubric}
\begin{rubric}
	Then he shall uncover the Cup, genuflect, collect the fragments (if there be any), and cleanse the Paten over the Cup, saying meanwhile:
\end{rubric}
\par\noindent
What reward shall I give unto the Lord for all the benefits that he hath done unto me? I will receive the cup of salvation, and call upon the Name of the Lord. I will call upon the Lord which is worthy to be praised, so shall I be safe from mine enemies.
\begin{rubric}
    Taking the Cup in his right hand and signing himself with it, he saith,
\end{rubric}\par\noindent
The Blood of our Lord Jesus Christ preserve my soul unto everlasting life. Amen.
\begin{rubric}
    Holding the Paten under the Cup with his left hand, the Priest is to reverently receive the Blood with the particle.
\end{rubric}
\begin{rubric}
	Having received, if there be any to be communicated, let him communicate them before he purifieth himself.
\end{rubric}
\end{multicols}
\subsec{Communion of the People}
\begin{rubric}
    If there be any to be communicated, the Priest shall genuflect and place the consecrated particles in a Ciborium (or if there are few to be communicated, on the Paten), unless from the beginning they had been placed in a Ciborium.
\end{rubric}
\begin{rubric}
    If the Priest is to administer Communion from the reserved Sacrament, he openeth the Tabernacle, genuflecteth, taketh out the Ciborium, and placeth it upon the Corporal.
\end{rubric}
\begin{rubric}
    Then shall the Priest genuflect, take the Ciborium with his left hand (or the Cup, if he consecrated on the Paten), and take one particle with his right hand takes, which he holdeth between his thumb and forefinger slightly raised above the Ciborium (Cup); and turning to the people in the midst of the Altar, he saith in the clear voice:
\end{rubric}
\elcol{℣. Behold the Lamb of God. Behold him who taketh away the sins of the world.}{℣. Ecce Agnus Dei, ecce, qui tollit peccáta mundi.}
\elcol{℟. Lord, I am not worthy that thou shouldest come under my roof, but speak the word only, and my soul shall be healed.}{℟. Dómine, non sum dignus, ut intres sub tectum meum, sed tantum dic verbo, et sanábitur ánima mea.}
\elcol{℟. Lord, I am not worthy that thou shouldest come under my roof, but speak the word only, and my soul shall be healed.}{℟. Dómine, non sum dignus, ut intres sub tectum meum, sed tantum dic verbo, et sanábitur ánima mea.}
\elcol{℟. Lord, I am not worthy that thou shouldest come under my roof, but speak the word only, and my soul shall be healed.}{℟. Dómine, non sum dignus, ut intres sub tectum meum, sed tantum dic verbo, et sanábitur ánima mea.}
\begin{rubric}
%This rubric is adapted from the Antiochian Service Book (Eastern Rite).
	After the Priest returneth to the Altar, the Pre-Communion Hymn (p. \pageref{byzantine}) is to then be sung as communicants approach the Altar Rail.\par
	\textsc{Note,} The Pre-Communion Hymn may be said or sung any time after the Prayer of Humble Access.
\end{rubric}
\begin{rubric}\label{communionrubrics}
    Only Orthodox Christians in good standing and properly disposed may receive the Eucharist.\par
    \textsc{Note,} Each communicant shall receive the Communion kneeling.
\end{rubric}
\begin{rubric}
    If they are to communicate, the Priest first communicateth the Sacred Ministers, and then the other Priests and Clerics in choir. (Priest and Deacons shall wear a stole either of white colour or of the same colour as the administering Priest.) And last of all, he proceedeth to communicate the others, beginning with those on the Epistle side.
\end{rubric}
\begin{rubric}
    If the Body and Blood of Christ are to be administered by intinction, then the Priest, when giving the Sacrament to each one, intincteth the Host into the Cup; then, making with the Host the sign of the Cross over the Cup, he placeth the Host on the tongue of each communicant, while saying:
\end{rubric}
\needspace{4\baselineskip}
%Adapted the Proposed English 1928 BCP
\lett{T}{he} Body of our Lord Jesus Christ which was given for thee, and his Blood which was shed for thee. Take them in remembrance that Christ died for thee, and feed on him in thy heart by faith with thanksgiving.
\begin{rubric}
    But if both kinds are to be administered separately, the Priest, when giving the Body, shall make with the Host the sign of the Cross over the Ciborium and place the Host on the tongue of each communicant, while saying:
\end{rubric}
\needspace{4\baselineskip}
 \lett{T}{he} Body of our Lord Jesus Christ, which was given for thee, preserve thy body and soul unto everlasting life. Take and eat this in remembrance that Christ died for thee, and feed on him in thy heart by faith, with thanksgiving.
\begin{rubric}
    And then, while administering the Cup to each communicant, the Priest, keeping hold of the Cup, shall say:
\end{rubric}
%\needspace{4\baselineskip}
\lett{T}{he} Blood of our Lord Jesus Christ, which was shed for thee, preserve thy body and soul unto everlasting life. Drink this in remembrance that Christ's Blood was shed for thee, and be thankful.
\begin{rubric}
    The Communion ended, the Priest shall return to the Altar, saying nothing, nor doth he give the Blessing since he will give it at the end of Mass.    
\end{rubric}
\begin{rubric}
    When the faithful have received Holy Communion, the Deacon and Priest return the Blessed Sacrament to the Tabernacle. The Priest shall then consume the Precious Blood in the Cup.\par
    \textsc{Note,} When prepared for reservation, the Host shall be touched with the Blood.
\end{rubric}
\begin{rubric}
    Then shall the Priest say,
\end{rubric}
\lett{G}{rant,} O Lord, that what we have taken with our mouths we may receive with a pure heart: and from a temporal gift may it become to us an everlasting remedy.
\begin{rubric}
    Meanwhile, he presenteth the Cup to the Minister, who shall pour in a little wine, wherewith he purifieth himself. Then he continueth,    
\end{rubric}
\lett{L}{et} thy Body, O Lord, which I have taken, and thy Blood which I have drunk, cleave to my members: and grant; that no stain of sin may remain in me, whom thou hast refreshed with pure and holy sacraments: Who livest and reignest, world without end. Amen.
\begin{rubric}
    Then shall the Priest wash and wipe his fingers and take the ablution. Then he wipeth his mouth and the Cup.
\end{rubric}
\begin{rubric}
	After folding the Corporal, he is to cover the Cup and place it upon the Altar as before. Then he proceedeth with the Mass.
\end{rubric}
\subby{Thanksgiving}
\begin{rubric}
    Standing with his hands joined, the Priest shall read the Communion Antiphon. Then, at the Epistle corner, he alone saith the Thanksgiving.
\end{rubric}
\letuspray
\lett{A}{lmighty} and everliving God, we most heartily thank thee, for that thou dost vouchsafe to feed us who have duly received these holy mysteries with the spiritual food of the most precious Body and Blood of thy Son our Saviour Jesus Christ; and dost assure us thereby of thy favour and goodness towards us; and that we are very members incorporate in the mystical body of thy Son, which is the blessed company of all faithful people; and are also heirs through hope of thy everlasting kingdom, by the merits of his most precious death and passion. And we humbly beseech thee, O heavenly Father, so to assist us with thy grace, that we may continue in that holy fellowship, and do all such good works as thou hast prepared for us to walk in; through Jesus Christ our Lord, to whom, with thee and the Holy Ghost, be all honour and glory, world without end.\par
℟. Amen.

\subsec{Dismissal}

\begin{rubric}
    Then, with hands joined before his breast, the Priest shall proceed to the midst of the Altar, kiss the Altar, and turn to the People, with hands extended, saying:
\end{rubric}
\elcol{℣. The Lord be with you.}{℣. Dóminus vobíscum.}
\elcol{℟. And with thy spirit.}{℟. Et cum spíritu tuo.}
\begin{rubric}
    Then, turning back to the Book, he saith:
\end{rubric}
\elcol{℣. Let us pray.}{℣. Orémus.}
\begin{rubric}
The Priest then saith the Postcommunion Prayers in the same manner, number, and order as the Collects at the beginning of Mass.
\end{rubric}
\begin{rubric}
    After saying the last Prayer, the Priest is to return to the midst of the Altar, kissing it, and turning toward the people, saying:
\end{rubric}
\elcol{℣. The Lord be with you.}{℣. Dóminus vobíscum.}
\elcol{℟. And with thy spirit.}{℟. Et cum spíritu tuo.}
    \begin{rubric}
    	{On days when the \emph{Gloria in exclesis} is said,}
    \end{rubric}
    \elcol{℣. Depart in peace.}{℣. Ite, Missa est.}
    \elcol{℟. Thanks be to God.}{℟. Deo grátias.}
    \begin{rubric}
    	{On days when the \emph{Gloria in excelsis} is not said,}
    \end{rubric}
    \elcol{℣. Let us bless the Lord.}{℣. Benedicámus Dómino.}
    \elcol{℟. Thanks be to God.}{℟. Deo grátias.}
\begin{rubric}
In Requiem Masses,
\end{rubric}
    \elcol{℣. May they rest in peace.}{℣. Requiéscant in pace.}
    \elcol{℟. Amen.}{℟. Amen.}
\begin{rubric}
    In Eastertide, in Masses of the Season,
\end{rubric}
    \elcol{℣. Depart in peace, alleluia, alleluia.}{℣. Ite, Missa est, allelúja, allelúja.}
    \elcol{℟. Thanks be to God, alleluia, alleluia.}{℟. Deo grátias, allelúja, allelúja.}
\begin{rubric}
    Having said the Dismissal, the Priest boweth before the midst of the Altar, and with hands joined thereon, saith in the secret voice: \emph{Let this my bounden duty}.
\end{rubric}
\begin{rubric}
    Then---unless it be a Requiem Mass---the Congregation kneeling, the Priest shall kiss the Altar; raise his eyes; extend, raise, and join his hands; bow his head to the Cross; and give the Blessing, saying,
\end{rubric}
\lett{T}{he} Peace of God, which passeth all understanding, keep your hearts and minds in the knowledge and love of God, and of his Son Jesus Christ our Lord: And the Blessing of God Almighty,\margrub{He turns to the people, blessing them once only, even in solemn Masses.} the Father, {\ding{64}} the Son, and the Holy Ghost, be amongst you, and remain with you always. \textit{Amen.}
\begin{rubric}
    In Pontifical Masses, the Blessing is given by the Bishop and is threefold, as ordered in the Pontifical.
\end{rubric}
\begin{rubric}
    In Masses of the Dead, the Blessing is not given, but having said \emph{Requiéscant in pace} and \emph{Let this my bounden duty}, he kisseth the Altar and readeth the Last Gospel.
\end{rubric}
\subsec{Last Gospel}
\begin{rubric}
    Unless a proper Last Gospel be provided, the Priest shall conclude the Mass with this Last Gospel, at the Gospel corner and with hands joined:
\end{rubric}
\elcol{℣. The Lord be with you.}{℣. Dóminus vobíscum.}
\elcol{℟. And with thy spirit.}{℟. Et cum spíritu tuo.}
\elcol{℣. The\margcolrub{He signs with the sign of the Cross first the Altar or the Book, then himself on the forehead, mouth, \& breast.} {\ding{66}} beginning of the Holy Gospel according to John.}{℣. Inítium\marglatincolrub{He signs with the sign of the Cross first the Altar or the Book, then himself on the forehead, mouth, \& breast.} {\ding{66}} sancti Evangélii secúndum Joánnem}
\elcol{℟. Glory be to thee, O Lord.}{℟. Glória tibi, Dómine.}
\elcol{In the beginning was the Word, and the Word was with God, and the Word was God. The same was in the beginning with God. All things were made by him; and without him was not any thing made that was made. In him was life; and the life was the light of men. And the light shineth in darkness; and the darkness comprehended it not. There was a man sent from God, whose name was John. The same came for a witness, to bear witness of the Light, that all men through him might believe. He was not that Light, but was sent to bear witness of that Light. That was the true Light, which lighteth every man that cometh into the world. He was in the world, and the world was made by him, and the world knew him not. He came unto his own, and his own received him not. But as many as received him, to them gave he power to become the sons of God, even to them that believe on his name: Which were born, not of blood, nor of the will of the flesh, nor of the will of man, but of God. \inrub{Everyone genuflects.} And the Word was made flesh, and dwelt among us, \inrub{Everyone rises.} (and we beheld his glory, the glory as of the only begotten of the Father,) full of grace and truth.}{In princípio erat Verbum, et Verbum erat apud Deum, et Deus erat Verbum. Hoc erat in princípio apud Deum. Omnia per ipsum facta sunt: et sine ipso factum est nihil, quod factum est: in ipso vita erat, et vita erat lux hóminum: et lux in ténebris lucet, et ténebræ eam non comprehendérunt.
Fuit homo missus a Deo, cui nomen erat Joánnes. Hic venit in testimónium, ut testimónium perhibéret de lúmine, ut omnes créderent per illum. Non erat ille lux, sed ut testimónium perhibéret de lúmine.
Erat lux vera, quæ illúminat omnem hóminem veniéntem in hunc mundum. In mundo erat, et mundus per ipsum factus est, et mundus eum non cognóvit. In própria venit, et sui eum non recepérunt. Quotquot autem recepérunt eum, dedit eis potestátem fílios Dei fíeri, his, qui credunt in nómine ejus: qui non ex sanguínibus, neque ex voluntáte carnis, neque ex voluntáte viri, sed ex Deo nati sunt. \inrub{Everyone genuflects.} Et Verbum caro factum est, \inrub{Everyone rises.} et habitávit in nobis: et vídimus glóriam ejus, glóriam quasi Unigéniti a Patre, plenum gráti{\ae} et veritátis.}
\elcol{℟. Thanks be to God.}{℟. Deo grátias.}
\begin{rubric}
    At High Mass, a hymn may be sung while the Priest and servers leave the sanctuary.
\end{rubric}
\begin{rubric}
    As he departeth from the Altar, the Priest saith for thanksgiving the Antiphon \emph{Let us sing}, with the rest (p. \pageref{CommunionThanksgiving}).
\end{rubric}
\subsec{General Rubrics}\label{GeneralRubrics}
%The Book of Common Prayer has always contained rubrics at the beginning and end of the liturgical text in order for the good order of the Church. Due to the unique situation of the Vicariate, it is fitting that these be accommodated to our use.
%\begin{rubric}
%Notwithstanding anything that is elsewhere enjoined in any Rubrick or Canon, the Priest, in celebrating the Holy Communion, shall wear the vestments proper to the Priest as of 1950 in dissident Rome. That is, Amice, Alb, Cincture, Maniple, Stole, and Chasuble (the latter three of the colour of the day).
%\end{rubric}
\begin{rubric}
    It is expedient for the Priest to have, set up on the Altar, cards with the most common prayers (or at least a convenient pamphlet), including the those only mentioned by their incipit in this Book.\par
\end{rubric}
%\begin{rubric}
%The Priest, the presence of an Ordinary notwithstanding, when he processes or reposes, wears a Canterbury Cap or Biretta.
%\end{rubric}
%\begin{rubric}
%There shall be no celebration of the Lord's Supper without at least one person, other than the Priest, physically present.
%\end{rubric}
\begin{rubric}
The Altar, at the Communion-time---having a fair white linen cloth, a cross, and at least two candles upon it---shall stand in the Sanctuary.
\end{rubric}
\begin{rubric}
The Bread for the Eucharist ought to be provided by the Priest or the members of the Congregation. It must be leavened. That is, before and after being initially baked, it must be made of---and only made of---wheat flour, water, yeast, and salt. The Wine should be made of grapes, without any additives or supplements. The Cup for the Eucharist should be made of precious metal.
\end{rubric}
\begin{rubric}
It is an ancient and laudable rule of the Church to receive this Holy Sacrament fasting. That is, one who desireth to receive the Eucharist must consume neither food nor drink (water excepted) from the midnight before the liturgy until the time of reception. In the case of Masses which occur after noon, it is sufficient that the fasting begin at noon or immediately after one's noon-time meal.
\end{rubric}

%\begin{center}
%	{\textsc{Here Endeth the Liturgy of St. Tikhon.}}
%\end{center}
%\clearpage
%\subby{Exhortations for Holy Communion}
%\begin{rubric}
%	At the time of the celebration of the Communion, the Priest may say this Exhortation. And Note, that the Exhortation shall be said on the First Sunday in Advent, the First Sunday in Lent, and Trinity Sunday.
%\end{rubric}
%\lett{D}{early} beloved in the Lord, ye who mind to come to the holy Communion of the Body and Blood of our Saviour Christ, must consider how Saint Paul exhorteth all persons diligently to try and examine themselves, before they presume to eat of that Bread, and drink of that Cup. For as the benefit is great, if with a true penitent heart and lively faith we receive that holy Sacrament; 
%Addition:
%that is, the very and true Body and Blood of Our Lord Jesus Christ; 
%
%so is the danger great, if we receive the same unworthily. Judge therefore yourselves, brethren, that ye be not judged of the Lord; repent you truly for your sins past; 
%Addition:
%confess them to your spiritual father; 
%
%have a lively and steadfast faith in Christ our Saviour; amend your lives, and be in perfect charity with all men; so shall ye be meet partakers of those holy mysteries. And above all things ye must give most humble and hearty thanks to God, the Father, the Son, and the Holy Ghost, for the redemption of the world by the death and passion of our Saviour Christ, both God and man; who did humble himself, even to the death upon the Cross, for us, miserable sinners, who lay in darkness and the shadow of death; that he might make us the children of God, and exalt us to everlasting life. And to the end that we should always remember the exceeding great love of our Master, and only Saviour, Jesus Christ, thus dying for us, and the innumerable benefits which by his precious blood-shedding he hath obtained for us; he hath instituted and ordained holy mysteries, as pledges of his love, and for a continual remembrance of his death, to our great and endless comfort. To him therefore, with the Father and the Holy Ghost, let us give (as we are most bounden) continual thanks; submitting ourselves wholly to his holy will and pleasure, and studying to serve him in true holiness and righteousness all the days of our life. Amen.
%\begin{rubric}
%	When the Minister giveth warning for the Celebration of the Holy Communion, (which he shall always do upon the Sunday, or some Holy-day, immediately preceding,) he shall read this Exhortation following, or so much thereof as, in his discretion, he may think convenient.
%\end{rubric}
%\lett{D}{early} beloved, on \textit{N.} day next I purpose, through God's assistance, to administer to all such as shall be religiously and devoutly disposed the most comfortable Sacrament of the Body and Blood of Christ; to be by them received in remembrance of his meritorious Cross and Pas-sion; whereby alone we obtain remission of our sins, and are made partakers of the Kingdom of heaven. Wherefore it is our duty to render most humble and hearty thanks to Almighty God, our heavenly Father, for that he hath given his Son our Saviour Jesus Christ, not only to die for us, but also to be our spiritual food and sustenance in that holy Sacrament. Which being so divine and comfortable a thing to them who receive it worthily, and so dangerous to those who will presume to receive it unworthily; my duty is to exhort you, in the mean season to consider the dignity of that holy mystery, and the great peril of the unworthy receiving thereof; and so to search and examine your own consciences, and that not lightly, and after the manner of dissemblers with God; but so that ye may come holy and clean to such a heavenly Feast, in the marriage-garment required by God in holy Scripture, and be received as worthy partakers of that holy Table.
%
%    The way and means thereto is: First, to examine your lives and conversations by the rule of God's commandments; and whereinsoever ye shall perceive yourselves to have offended, either by will, word, or deed, there to bewail your own sinfulness, and to confess yourselves to Almighty God, with full purpose of amendment of life. And if ye shall perceive your offences to be such as are not only against God, but also against your neighbours; then ye shall reconcile yourselves unto them; being ready to make restitution and satisfaction, according to the uttermost of your powers, for all injuries and wrongs done by you to any other; and being likewise ready to forgive others who have offended you, as ye would have forgiveness of your offences at God's hand: for otherwise the receiving of the holy Communion doth nothing else but increase your condemnation. Therefore, if any of you be a blasphemer of God, an hinderer or slanderer of his Word, an adulterer, or be in malice, or envy, or in any other grievous crime; repent you of your sins, or else come not to that holy Table.
%    
%    And because it is requisite that no man should come to the holy Communion, but with a full trust in God's mercy, and with a quiet conscience; 
    %This phrase is omitted so to indicate that all such sinners ought to confess:
    %therefore, if there be any of you, who by this means cannot quiet his own conscience herein, but requireth further comfort or counsel, 
%    let him come to me, or to some other Priest of 
    %Change:
%    Christ's Church, confessing his sins and opening his grief; that he may receive such godly counsel, advice, and absolution of his sins as may tend to the quieting of his conscience, the removing of all scruple and doubtfulness, and the sure benefit of Jesus Christ's promise: the remission of sins and bestowing of the benefits of his Passion.
%\begin{rubric}
%	Or, in case he shall see the People negligent to come to the Holy Communion, instead of the former, he may use this Exhortation.
%\end{rubric}
%\lett{D}{early} beloved brethren, on \textit{N.} I intend, by God's grace, to celebrate the 
%Change:
%Holy Sacrifice of the Mass: 
%
%unto which, in God's behalf, I bid you all who are here present; and beseech you, for the Lord Jesus Christ's sake, that ye will not refuse to come thereto, being so lovingly called and bidden by God himself. Ye know how grievous and unkind a thing it is, when a man hath prepared a rich feast, decked his table with all kind of provision, so that there lacketh nothing but the guests to sit down; and yet they who are called, without any cause, most unthankfully refuse to come. Which of you in such a case would not be moved? Who would not think a great injury and wrong done unto him? Wherefore, most dearly beloved in Christ, take ye good heed, lest ye, withdrawing yourselves from this holy Supper, provoke God's indignation against you. It is an easy matter for a man to say, I will not communicate, because I am otherwise hindered with worldly business. But such excuses are not so easily accepted and allowed before God. If any man say, I am a grievous sinner, and therefore am afraid to come: wherefore then do ye not repent and amend? 
%Addition:
%Wherefore do ye not approach myself or one of Christ's Priests for Confession and Absolution? 
%
%When God calleth you, are ye not ashamed to say ye will not come? When ye should return to God, will ye excuse yourselves, and say ye are not ready? Consider earnestly with yourselves how little such feigned excuses will avail before God. Those who refused the feast in the Gospel, because they had bought a farm, or would try their yokes of oxen, or because they were married, were not so excused, but counted unworthy of the heavenly feast. Wherefore, according to mine office, I bid you in the Name of God, I call you in Christ's behalf, I exhort you, as ye love your own salvation, that ye will be partakers of this holy Communion. And as the Son of God did vouchsafe to yield up his soul by death upon the Cross for your salvation; so it is your duty to receive the Communion in remembrance of the sacrifice of his death, as he himself hath commanded: which if ye shall neglect to do, consider with yourselves how great is your ingratitude to God, and how sore punishment hangeth over your heads for the same; when ye wilfully abstain from the Lord's Table, and separate from your brethren, who come to feed on the banquet of that most heavenly food. These things if ye earnestly consider, ye will by God’s grace return to a better mind: for the obtaining whereof we shall not cease to make our humble petitions unto Almighty God, our heavenly Father.