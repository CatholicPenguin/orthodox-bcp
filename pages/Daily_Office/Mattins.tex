\fancyhead[RO,LE]{\textit{Mattins}}
\section{The Order for Daily Morning Prayer}
\elcol{℣. O Lord, {\ding{61}} open thou our lips.}{℣. Dómine, {\ding{61}} lábia nostra apéries.}
\elcol{℟. And our mouth shall show forth thy praise.}{℟. Et os nostrum annuntiábit laudem tuam.}
\elcol{℣. O God, {\ding{64}} make speed to save us.}{℣. Deus, {\ding{64}} in adjutórium nostrum inténde.}
\elcol{℟. O Lord, make haste to help us.}{℟. Dómine, ad adjuvándum nos festína.}
\begin{rubric}
    Here, all standing up, the Minister shall say,
\end{rubric}
\elcol{℣. Glory be to the Father, and to the Son, and to the Holy Ghost.}{℣. Glória Patri, et Fílio, * et Spirítui Sancto:}
\elcol{℟. As it was in the beginning, is now, and ever shall be, world without end. Amen.}{℟. Sicut erat in princípio, et nunc, et semper, * et in sǽcula s{\ae}culórum. Amen.}
\elcol{℣. Praise ye the Lord.}{℣. Laudáte Dominum.}
\elcol{℟. The Lord's Name be praised.}{℟. Sit Nomen Dómini Benedíctum.}
\begin{rubric}
    Then shall be said or sung the following Psalm; except on those days for which an Anthem is appointed; and except also, when Psalm 95 may occur in the course of the Psalms.
\end{rubric}
\begin{rubric}
    But \textsc{Note}, That on Ash Wednesday and Good Friday the \emph{Venite} may be omitted.
\end{rubric}
\begin{rubric}
    On the days hereafter named, immediately before and after the \emph{Venite} may be sung or said,
\end{rubric}
\par\noindent
\inv{Advent:} Our King and Saviour draweth nigh; * O come, let us adore him.
\par\noindent
\inv{Christmastide:} Alleluia. Unto us a child is born; * O come, let us adore him. Alleluia.
\par\noindent
\inv{Epiphanytide \& Transfiguration:} The Lord hath manifested forth his glory; * O come, let us adore him.
\par\noindent
%Adapted from AOB:
\inv{Septuagesimatide:} Let us come before the presence of the Lord with thanksgiving;  * O come, let us adore him.
\par\noindent
%Adapted from the 1962 BCP:
\inv{Lent:} The goodness of God leadeth to repentance; * O come, let us adore him.
\par\noindent
\inv{Passiontide:} Christ our Lord became obedient unto death; * O come, let us adore him.
\par\noindent
\inv{Eastertide outside the Octave:} Alleluia. The Lord is risen indeed; * O come let us adore him. Alleluia.
\par\noindent
\inv{Ascensiontide:} Alleluia. Christ the Lord ascended into heaven; * O come, let us adore him. Alleluia.
\par\noindent
\inv{Whitsuntide:} Alleluia. The Spirit of the Lord filleth the world; * O come, let us adore him. Alleluia.
\par\noindent
\inv{Trinitytide:} Father, Son, and Holy Ghost, One God; * O come, let us adore him.
\par\noindent
\inv{Feasts of Our Lord \& Our Lady:} The Word was made flesh; * O come, let us adore him.
\par\noindent
%Due to the wider use of Feast Days in the AWRV, it is difficult to decide how to adapt this rubric (Feast Days which have a Proper Epistle and Gospel). Given that Semidouble Feast Days almost never occur, and moving the bar so low to Simplex or Memorial would destroy the desired effect, Semidouble seems to be the best rank. For it would not cast too wide of a net: effectively capturing all Doubles and their Octaves, except when the Octave is meant to not be noticed intensely.
\inv{Other Feast Days of Semidouble or higher:} The Lord is glorious in his saints; * O come, let us adore him.
%For text of the psalms, the Latin Monastic Diurnal is used.
\subby{Venite, exultemus Domino.}
\elcol{\lett{O}{come,} let us sing unto the Lord : let us heartily rejoice in the strength of our salvation.\par
\secondlinepsalm{Let us come before his presence with thanksgiving : and shew ourselves glad in him with psalms.}
\thirdlinepsalm{For the Lord is a great God : and a great King above all gods.}
\psanum{4}In his hand are all the corners of the earth : and the strength of the hills is his also.\par
\psanum{5}The sea is his, and he made it : and his hands prepared the dry land.\par
\psanum{6}O come, let us worship and fall down : and kneel before the Lord our Maker.\par
\psanum{7}For he is the Lord our God : and we are the people of his pasture, and the sheep of his hand.\par
\psanum{8}To-day if ye will hear his voice, harden not your hearts : as in the provocation, and as in the day of temptation in the wilderness.\par
\psanum{9}When your fathers tempted me : proved me, and saw my works.\par
\psanum{10}Forty years long was I grieved with this generation, and said : It is a people that do err in their hearts, for they have not known my ways;\par
\psanum{11}Unto whom I sware in my wrath : that they should not enter into my rest.}
{\lett{V}{en\smash{í}te}, exsultémus Dómino, jubilémus Deo, salutári nostro:\par
\secondline{Pr{\ae}occupémus fáciem ejus in confessióne, et in psalmis jubilémus ei.}
\thirdline{Quóniam Deus magnus Dóminus, et Rex magnus super omnes deos, quóniam non repéllet Dóminus plebem suam:}
Quia in manu ejus sunt omnes fines terr{\ae}, et altitúdines móntium ipse cónspicit.\par
Quóniam ipsíus est mare, et ipse fecit illud, et áridam fundavérunt manus ejus\par
Veníte, adorémus, et procidámus ante Deum: plorémus coram Dómino, qui fecit nos,\par
Quia ipse est Dóminus, Deus noster; nos autem pópulus ejus, et oves páscu{\ae} ejus.\par
Hódie, si vocem ejus audiéritis, nolíte obduráre corda vestra, sicut in exacerbatióne secúndum diem tentatiónis in desérto: ubi tentavérunt me patres vestri, probavérunt et vidérunt ópera mea.
\par
Quadragínta annis próximus fui generatióni huic, et dixi; Semper hi errant corde, ipsi vero non cognovérunt vias meas: quibus jurávi in ira mea; Si introíbunt in réquiem meam.}

\begin{rubric}
    %1928 BCP Rubric (permission to not use Gloria Patri, except at the end of all of the Psalms, is removed):
    Then shall follow a Portion of the Psalms, according to the Use of this Church. And at the end of every Psalm, and likewise at the end of the \emph{Venite}, daily Old Testament Canticle, and \emph{Benedictus}, shall be sung or said the \emph{Gloria Patri}.
\end{rubric}
\elcol{℣. Glory be to the Father, and to the Son, * and to the Holy Ghost.\par
℟. As it was in the beginning, is now, and ever shall be, * world without end. Amen.}{℣. Glória Patri, et Fílio, * et Spirítui Sancto:\par
℟. Sicut erat in princípio, et nunc, et semper, * et in sǽcula s{\ae}culórum. Amen.}

\par\noindent
	\centerline{\rule{0.5\textwidth}{0.4pt}}
\par\noindent
\begin{rubric}
    Verses 8-11 of the \emph{Venite} may be replaced with the following.
\end{rubric}
\elcol{O worship the Lord in the beauty of holiness : let the whole earth stand in awe of him.\par
For he cometh, for he cometh to judge the earth : and with righteousness to judge the world, and the people with his truth.
}{Tóllite hóstias, et introíte in átria ejus: * adoráte Dóminum in átrio sancto ejus.\par
Judicábit orbem terr{\ae} in {\ae}quitáte, * et pópulos in veritáte sua.}
\par\noindent
	\centerline{\rule{0.5\textwidth}{0.4pt}}
\par\noindent
\begin{rubric}
%Rubric added to permit the Sarum Invitatory Psalm.
    After the \emph{Venite}, a Hymn may be sung.
\end{rubric}
\begin{rubric}
Then shall be read the First Lesson, according to the Table or Calendar. And \textsc{Note}, That before every Lesson, the Minister shall say, \emph{Here beginneth} such a \emph{Chapter} (or \emph{Verse of} such a \emph{Chapter}) \emph{of} such a Book; and after every Lesson, \emph{Here endeth the First} (or \emph{the Second Lesson.})
\end{rubric}
\vspace{-2ex}
\subby{Te Deum laudamus}
%Reversion to English language brings the Te Deum is conformity to the Monastic Matins use.
\begin{rubric}
%Great reduction of the Office, allowed in the 1928, not reproduced. Additional Canticles supplied.
The \emph{Te Deum} is prayed on all Sundays (outside Advent \& Lent) and Feast Days %This rubric brings the Te Deum to the same standard as the other Festal canticles and brings it closer to Breviary use.
(Semidouble or higher).\par
\textsc{Note,} The Old Testament Canticle (p. \pageref{OT}) may be said instead of the \emph{Te Deum}.
\end{rubric}
\elcol{
\lett{W}{e} praise thee, O God; we acknowledge thee to be the Lord.\par
\secondline{All the earth doth worship thee, the Father everlasting.}
\thirdline{To thee all Angels cry aloud; the Heavens, and all the Powers therein;}
    To thee Cherubim and Seraphim continually do cry,\par
    Holy, Holy, Holy, Lord God of Sabaoth;\par
    Heaven and earth are full of the Majesty of thy glory.\par
    The glorious company of the Apostles praise thee.\par
    The goodly fellowship of the Prophets praise thee.\par
    The noble army of Martyrs praise thee.\par
    The holy Church throughout all the world doth acknowledge thee;\par
    The Father of an infinite Majesty;\par
    %Adorable reverted to honourable in line with Prayer Book Tradition and the Latin text.
    Thine honourable, true and only Son;\par
    Also the Holy Ghost the Comforter.
    
\lett{T}{hou} art the King of Glory, O Christ.\par
\secondline{Thou art the everlasting Son of the Father.}
%`thou didst humble thyself to be born of a Virgin' reverted to `thou didst not abhor the Virgin’s womb' in line with the Prayer Book Tradition and Latin text.
\thirdline{When thou tookest upon thee to deliver man, thou didst not abhor the Virgin’s womb.}
    When thou hadst overcome the sharpness of death, thou didst open the Kingdom of Heaven to all believers.\par
    Thou sittest at the right hand of God, in the glory of the Father.\par
    We believe that thou shalt come to be our Judge.\par
    We therefore pray thee, help thy servants, whom thou hast redeemed with thy precious blood.\par
    Make them to be numbered with thy Saints, in glory everlasting.
    
\lett{O}{Lord,} save thy people, and bless thine heritage.\par
\secondline{Govern them and lift them up for ever.}
\thirdline{Day by day we magnify thee;}
    And we worship thy Name ever, world without end.\par
    Vouchsafe, O Lord, to keep us this day without sin.\par
    O Lord, have mercy upon us, have mercy upon us.\par
    %Be reverted to lightened in line with Prayer Book Tradition.
    O Lord, let thy mercy lighten upon us, as our trust is in thee.\par
    O Lord, in thee have I trusted; let me never be confounded.}
{\lett{T}{e} Deum laudámus: * te Dóminum confitémur.\par
\secondline{Te {\ae}térnum Patrem * omnis terra venerátur.}
\thirdline{Tibi omnes Ángeli, * tibi C{\ae}li, et univérs{\ae} Potestátes}
Tibi Chérubim et Séraphim * incessábili voce proclámant:\par
Sanctus, Sanctus, Sanctus * Dóminus Deus Sábaoth.\par
Pleni sunt c{\ae}li et terra * majestátis glóri{\ae} tu{\ae}.\par
Te gloriósus * Apostolórum chorus,\par
Te Prophetárum * laudábilis númerus,\par
Te Mártyrum candidátus * laudat exércitus.\par
Te per orbem terrárum * sancta confitétur Ecclésia,\par
Patrem * imméns{\ae} majestátis;\par
Venerándum tuum verum * et únicum Fílium;\par
Sanctum quoque * Paráclitum Spíritum.

\lett{T}{u} Rex glóri{\ae}, * Christe.\par
\secondline{Tu Patris * sempitérnus es Fílius.}
\thirdline{Tu, ad liberándum susceptúrus hóminem: * non horruísti Vírginis úterum.}
Tu, devícto mortis acúleo, * aperuísti credéntibus regna c{\ae}lórum.\par
Tu ad déxteram Dei sedes, * in glória Patris.\par
Judex créderis * esse ventúrus. \par
Te ergo quǽsumus, tuis fámulis súbveni, * quos pretióso sánguine redemísti.\par
{\AE}térna fac cum Sanctis tuis * in glória numerári.\par
\lett{S}{alvum} fac pópulum tuum, Dómine, * et bénedic hereditáti tu{\ae}.\par
\secondline{Et rege eos, * et extólle illos usque in {\ae}térnum.}
\thirdline{Per síngulos dies * benedícimus te.}
Et laudámus nomen tuum in sǽculum, * et in sǽculum sǽculi.\par
Dignáre, Dómine, die isto * sine peccáto nos custodíre.\par
Miserére nostri, Dómine, * miserére nostri.\par
Fiat misericórdia tua, Dómine, super nos, * quemádmodum sperávimus in te.\par
In te, Dómine, sperávi: * non confúndar in {\ae}térnum.}

%MANUAL ADJUSTMENT:
\vspace{-2ex}
\subby{Benedicite, omnia opera Domini.}

%MANUAL ADJUSTMENT:
\vspace{-1ex}

\begin{rubric}
The \emph{Benedicite} is prayed on all Sundays in Advent \& Lent, and on Days below Semidouble.\par
\textsc{Note,} The Old Testament Canticle (p. \pageref{OT}) may be said instead of the \emph{Benedicite}.
\end{rubric}
\elcol{\lett{O}{all} ye Works of the Lord, bless ye the Lord: * praise him, and magnify him for ever.\par
\secondline{O ye Angels of the Lord, bless ye the Lord: * praise him, and magnify him for ever.}

\lett{O}{ye} Heavens, bless ye the Lord: * praise him, and magnify him for ever.\par
\secondline{O ye Waters that be above the firmament, bless ye the Lord: * praise him, and magnify him for ever.}
\thirdline{O all ye Powers of the Lord, bless ye the Lord: * praise him, and magnify him for ever.}
    O ye Sun and Moon, bless ye the Lord: * praise him, and magnify him for ever.\par
    O ye Stars of heaven, bless ye the Lord: * praise him, and magnify him for ever.\par
    O ye Showers and Dew, bless ye the Lord: * praise him, and magnify him for ever.\par
    O ye winds of God, bless ye the Lord: * praise him, and magnify him for ever.\par
    O ye Fire and Heat, bless ye the Lord * praise him, and magnify him for ever.\par
    O ye Winter and Summer, bless ye the Lord: * praise him, and magnify him for ever.\par
    O ye Dews and Frosts, bless ye the Lord: * praise him, and magnify him for ever.\par
    O ye Frost and Cold, bless ye the Lord: * praise him, and magnify him for ever.\par
    O ye Ice and Snow, bless ye the Lord * praise him, and magnify him for ever.\par
    O ye Nights and Days, bless ye the Lord: * praise him, and magnify him for ever.\par
    O ye Light and Darkness, bless ye the Lord: * praise him, and magnify him for ever.\par
    O ye Lightnings and Clouds, bless ye the Lord * praise him, and magnify him for ever.\par

\lett{O}{let} the Earth bless the Lord: * yea, let it praise him, and magnify him for ever.\par
\secondline{O ye Mountains and Hills, bless ye the Lord: * praise him, and magnify him for ever.}
\thirdline{O all ye Green Things upon the earth, bless ye the Lord: * praise him, and magnify him for ever.}
    O ye Wells, bless ye the Lord: * praise him, and magnify him for ever.\par
    O ye Seas and Floods, bless ye the Lord: * praise him, and magnify him for ever.\par
    O ye Whales, and all that move in the waters, bless ye the Lord: * praise him, and magnify him for ever.\par
    O all ye Fowls of the air, bless ye the Lord: * praise him, and magnify him for ever.\par
    O all ye Beasts and Cattle, bless ye the Lord: * praise him, and magnify him for ever.\par
    O ye Children of Men, bless ye the Lord: * praise him, and magnify him for ever.
    
\lett{O}{let} Israel bless the Lord: * praise him, and magnify him for ever.\par
\secondline{O ye Priests of the Lord, bless ye the Lord: * praise him, and magnify him for ever.}
\thirdline{O ye Servants of the Lord, bless ye the Lord: * praise him, and magnify him for ever.}
    O ye Spirits and Souls of the Righteous, bless ye the Lord: * praise him, and magnify him for ever.\par
    O ye holy and humble Men of heart, bless ye the Lord: * praise him, and magnify him for ever.

    \lett{L}{et} us bless the Father, and the Son, and the Holy Ghost: * praise him, and magnify him for ever.
    }{
\lett{B}{ened\smash{í}cite,} ómnia ópera Dómini, Dómino: * laudáte et superexaltáte eum in sǽcula.\par
\secondline{Benedícite, Ángeli Dómini, Dómino: * laudáte et superexaltáte eum in sǽcula.}
\lett{B}{ened\smash{í}cite,} c{\ae}li, Dómino:  * laudáte et superexaltáte eum in sǽcula.\par
\secondline{Benedícite, aqu{\ae} omnes, qu{\ae} super c{\ae}los sunt, Dómino: * laudáte et superexaltáte eum in sǽcula.}
\thirdline{Benedícite, omnes virtútes Dómini, Dómino: * laudáte et superexaltáte eum in sǽcula.}
Benedícite, sol et luna, Dómino: * laudáte et superexaltáte eum in sǽcula.\par
Benedícite, stell{\ae} c{\ae}li, Dómino: * laudáte et superexaltáte eum in sǽcula.\par
Benedícite, omnis imber et ros, Dómino: * laudáte et superexaltáte eum in sǽcula.\par
Benedícite, omnes spíritus Dei, Dómino: * laudáte et superexaltáte eum in sǽcula.\par
Benedícite, ignis et {\ae}stus, Dómino: * laudáte et superexaltáte eum in sǽcula.\par
Benedícite, frigus et {\ae}stus, Dómino: * laudáte et superexaltáte eum in sǽcula.\par
Benedícite, rores et pruína, Dómino: * laudáte et superexaltáte eum in sǽcula.\par
Benedícite, gelu et frigus, Dómino: * laudáte et superexaltáte eum in sǽcula.\par
Benedícite, glácies et nives, Dómino: * laudáte et superexaltáte eum in sǽcula.\par
Benedícite, noctes et dies, Dómino: * laudáte et superexaltáte eum in sǽcula.\par
Benedícite, lux et ténebr{\ae}, Dómino: * laudáte et superexaltáte eum in sǽcula.\par
Benedícite, fúlgura et nubes, Dómino: * laudáte et superexaltáte eum in sǽcula.\par

\lett{B}{ened\smash{í}cat} terra Dóminum: * laudet et superexáltet eum in sǽcula.\par
\secondline{Benedícite, montes et colles, Dómino: * laudáte et superexaltáte eum in sǽcula.}
\thirdline{Benedícite, univérsa germinántia in terra, Dómino: * laudáte et superexaltáte eum in sǽcula.}
Benedícite, fontes, Dómino: * laudáte et superexaltáte eum in sǽcula.\par
Benedícite, mária et flúmina, Dómino: * laudáte et superexaltáte eum in sǽcula.\par
Benedícite, cete, et ómnia, qu{\ae} movéntur in aquis, Dómino: * laudáte et superexaltáte eum in sǽcula.\par
Benedícite, omnes vólucres c{\ae}li, Dómino: * laudáte et superexaltáte eum in sǽcula.\par
Benedícite, omnes bésti{\ae} et pécora, Dómino: * laudáte et superexaltáte eum in sǽcula.\par
Benedícite, fílii hóminum, Dómino: * laudáte et superexaltáte eum in sǽcula.\par
\lett{B}{ened\smash{í}cat} Israël Dóminum: * laudet et superexáltet eum in sǽcula.\par
\secondline{Benedícite, sacerdótes Dómini, Dómino: * laudáte et superexaltáte eum in sǽcula.}
\thirdline{Benedícite, servi Dómini, Dómino: * laudáte et superexaltáte eum in sǽcula.}
Benedícite, spíritus, et ánim{\ae} justórum, Dómino: * laudáte et superexaltáte eum in sǽcula.\par
Benedícite, sancti, et húmiles corde, Dómino: * laudáte et superexaltáte eum in sǽcula.\par
\lett{B}{enedic\smash{á}mus} Patrem et Fílium cum Sancto Spíritu: * laudémus et superexaltémus eum in sǽcula.}
\begin{rubric}
    {Then shall be read, in like manner, the Second Lesson, taken out of the New Testament, according to the Table or Calendar.}
\end{rubric}
%\begin{rubric}
%    And after that may be sung a Hymn, then shall be sung or said the Hymn following.
%\end{rubric}

%Permission to shorten Benedictus removed.
\subby{Benedictus}\label{Benedictus}

%MANUAL ADJUSTMENT:
\vspace{-2ex}

\begin{rubric}
%Permission for shorter Office removed.
    After which may be sung or said a Hymn and then shall be sung or said the \emph{Benedictus}, as followeth.
\end{rubric}

%MANUAL ADJUSTMENT:
\vspace{-1ex}

\elcol{\lett{B}{lessed} {\ding{64}} be the Lord God of Israel; * for he hath visited and redeemed his people;\par
    \secondline{And hath raised up a mighty salvation for us, * in the house of his servant David;}
    \thirdline{As he spake by the mouth of his holy Prophets, * which have been since the world began;}
    That we should be saved from our enemies, * and from the hand of all that hate us.\par
    To perform the mercy promised to our forefathers, * and to remember his holy covenant;\par
    To perform the oath which he sware to our forefather Abraham, * that he would give us;\par
    That we being delivered out of the hand of our enemies * might serve him without fear;\par
    In holiness and righteousness before him, * all the days of our life.\par
    And thou, child, shalt be called the prophet of the Highest: * for thou shalt go before the face of the Lord to prepare his ways;\par
    To give knowledge of salvation unto his people * for the remission of their sins,\par
    Through the tender mercy of our God; * whereby the day-spring from on high hath visited us;\par
    To give light to them that sit in darkness, and in the shadow of death, * and to guide our feet into the way of peace.}
{\lett{B}{ened\smash{í}ctus} {\ding{64}} Dóminus, Deus Isra{\"e}l: * quia visitávit, et fecit redemptiónem plebi su{\ae}:\par
\secondline{Et eréxit cornu salútis nobis: * in domo David, púeri sui.}
\thirdline{Sicut locútus est per os sanctórum, * qui a sǽculo sunt, prophetárum ejus:}
Salútem ex inimícis nostris, * et de manu ómnium, qui odérunt nos.\par
Ad faciéndam misericórdiam cum pátribus nostris: * et memorári testaménti sui sancti.\par
Jusjurándum, quod jurávit ad Ábraham patrem nostrum, * datúrum se nobis:\par
Ut sine timóre, de manu inimicórum nostrórum liberáti, * serviámus illi.\par
In sanctitáte, et justítia coram ipso, * ómnibus diébus nostris.\par
Et tu, puer, Prophéta Altíssimi vocáberis: * pr{\ae}íbis enim ante fáciem Dómini, paráre vias ejus:\par
Ad dandam sciéntiam salútis plebi ejus: * in remissiónem peccatórum eórum:\par
Per víscera misericórdi{\ae} Dei nostri: * in quibus visitávit nos, óriens ex alto:\par
Illumináre his, qui in ténebris, et in umbra mortis sedent: * ad dirigéndos pedes nostros in viam pacis.}
%Permission for Psalm 100 removed.
\vspace{-2ex}
\subby{Apostles' Creed}
%While the Athanasian Creed was excised from the American tradition (during an attempt to remove every Creed), it is present in the Prayer Book Tradition, including the Canadian version. Therefore, it is thus supplied here. Since the Canadian tradition allows the Athanasian Creed on any day of the year, it is allowed likewise here.
\begin{rubric}
    Then shall be said the Apostles' Creed, by the Minister and the People, standing.\par%Permission for alternate words removed.
    \textsc{Note,} the Nicene Creed may be said instead of the Apostles' (p. \pageref{NiceneCreed}).
\end{rubric}
\begin{rubric}
    Upon these Feasts; \emph{Christmas Day}, the \emph{Epiphany}, Saint \emph{Matthias}, \emph{Easter Day}, \emph{Ascension Day}, \emph{Whitsunday}, Saint \emph{John Baptist}, Saint \emph{James}, Saint \emph{Bartholomew}, Saint \emph{Matthew}, Saint \emph{Simon} and Saint \emph{Jude}, Saint \emph{Andrew}, and upon \emph{Trinity Sunday}, shall be sung or said, instead of the Apostle's Creed, the Athanasian Creed (p. \pageref{Ath}).
\end{rubric}
\elcol{\lett{I}{believe} in God the Father Almighty, Maker of heaven and earth: And in Jesus Christ his only Son our Lord: Who was conceived by the Holy Ghost, Born of the Virgin Mary: Suffered under Pontius Pilate, Was crucified, dead, and buried: He descended into hell; The third day he rose again from the dead: He ascended into heaven, And sitteth on the right hand of God the Father Almighty: From thence he shall come to judge the quick and the dead.\par
    I believe in the Holy Ghost: The holy Catholic Church; The Communion of Saints: The Forgiveness of sins: The Resurrection of the body: {\ding{64}} And the Life everlasting. Amen.}
    {\lett{C}{redo} in Deum, Patrem omnipoténtem, Creatórem c{\ae}li et terr{\ae}. Et in Jesum Christum, Fílium ejus únicum, Dóminum nostrum: qui concéptus est de Spíritu Sancto, natus ex María Vírgine, passus sub Póntio Piláto, crucifíxus, mórtuus, et sepúltus: descéndit ad ínferos; tértia die resurréxit a mórtuis; ascéndit ad c{\ae}los; sedet ad déxteram Dei Patris omnipoténtis: inde ventúrus est judicáre vivos et mórtuos.\par
     Credo in Spíritum Sanctum, sanctam Ecclésiam cathólicam, Sanctórum communiónem, remissiónem peccatórum, carnis {\ding{64}} resurrectiónem, vitam {\ae}térnam. Amen.}

\begin{rubric}
    And after that, these Prayers following, the People devoutly kneeling; the Minister first pronouncing,
\end{rubric}
%\begin{rubric}
%    The Minister begins with the \emph{Dóminus vobíscum} only if he be in Major Orders. Otherwise, he shall use the second option.
%\end{rubric}
\elcol{℣. The Lord be with you.\par
\textit{or,} O Lord, hear our prayer.}{℣. Dóminus vobíscum.\par
\textit{vel,} Dómine, exáudi oratiónem nostram.}
\elcol{℟. And with thy spirit.\par
\textit{or,} And let our cry come unto thee.}{℟. Et cum spíritu tuo.\par
\textit{vel,} Et clamor noster ad te véniat.}
\elcol{℣. Let us pray.}{℣. Orémus.}
\elcol{℣. Lord, have mercy upon us.}{℣. Kýrie, eléison.}
\elcol{℟. Christ, have mercy upon us.}{℟. Christe, eléison.}
\elcol{℣. Lord, have mercy upon us.}{℣. Kýrie, eléison.}

\subby{Lord's Prayer}
\begin{rubric}
    Then the Minister, Clerks, and People---all kneeling---shall say the Lord's Prayer with a loud voice.
\end{rubric}
\elcol{\lett{O}{ur} Father, who art in heaven, Hallowed be thy Name. Thy kingdom come. Thy will be done on earth, As it is in heaven. Give us this day our daily bread. And forgive us our trespasses, As we forgive those who trespass against us. And lead us not into temptation; But deliver us from evil. Amen.}
{\lett{P}{ater} noster, qui es in c{\ae}lis, sanctificétur nomen tuum: advéniat regnum tuum: fiat volúntas tua, sicut in c{\ae}lo et in terra. Panem nostrum quotidiánum da nobis hódie: et dimítte nobis débita nostra, sicut et nos dimíttimus debitóribus nostris: et ne nos indúcas in tentatiónem: sed líbera nos a malo. Amen.}

\vspace{-0.5ex}
\subby{Preces}
\begin{rubric}
    Then the Minister standing up shall say,
\end{rubric}
\elcol{℣. O Lord, show thy mercy upon us.\par
℟. And grant us thy salvation.\par
℣. O Lord, save the \textit{State}.\par
℟. And mercifully hear us when we call upon thee.\par
℣. Endue thy Ministers with righteousness.\par
℟. And make thy chosen people joyful.\par
℣. O Lord, save thy people.\par
℟. And bless thine inheritance.\par
℣. Give peace in our time, O Lord.\par
℟. For it is thou, Lord, only, that makest us dwell in safety.\par
℣. O God, make clean our hearts within us.\par
℟. And take not thy Holy Spirit from us.}{℣. Osténde nobis, Dómine, misericórdiam tuam.\par
℟. Et salutáre tuum da nobis.\par
℣. Dómine salvam fac \textit{Civitatem}.\par
℟. Et exáudi nos cum invocámus te.\par
℣. Sacerdótes tui induántur Justítia.\par
℟. Et sancti tui exúltent.\par
℣. Salvum fac Pópulum tuum, Dómine.\par
℟. Et bénedic H{\ae}reditáti tu{\ae}.\par
℣. Da pacem Dómine in diébus nostris.\par
℟. Quóniam tu, Dómine, singuláriter in spe constituísti me.\par
℣. Cor mundum crea in nobis, O Deus.\par
℟. Et Spíritum Sanctum tuum ne áuferas a nobis.}
%\elcol{℣. O Lord, show thy mercy upon us.}{℣. Osténde nobis, Dómine, misericórdiam tuam.}
%\elcol{℟. And grant us thy salvation.}{℟. Et salutáre tuum da nobis.}
%\elcol{℣. O Lord, save the \textit{State}.}{℣. Dómine salvam fac \textit{Civitatem}.}
%\elcol{℟. And mercifully hear us when we call upon thee.}{℟. Et exáudi nos cum invocámus te.}
%\elcol{℣. Endue thy Ministers with righteousness.}{℣. Sacerdótes tui induántur Justítia.}
%\elcol{℟. And make thy chosen people joyful.}{℟. Et sancti tui exúltent.}
%\elcol{℣. O Lord, save thy people.}{℣. Salvum fac Pópulum tuum, Dómine.}
%\elcol{℟. And bless thine inheritance.}{℟. Et bénedic H{\ae}reditáti tu{\ae}.}
%\elcol{℣. Give peace in our time, O Lord.}{℣. Da pacem Dómine in diébus nostris.}
%See the notes for this petition in Evensong.
%\elcol{℟. Because there is none other that fighteth for us, but only thou, O God.}{℟. Quia non est álius qui pugnet pro nobis, nisi tu Deus noster.}
%Testing using American version.
%\elcol{℟. For it is thou, Lord, only, that makest us dwell in safety.}{℟. Quóniam tu, Dómine, singuláriter in spe constituísti me.}
%\elcol{℣. O God, make clean our hearts within us.}{℣. Cor mundum crea in nobis, O Deus.}
%\elcol{℟. And take not thy Holy Spirit from us.}{℟. Et Spíritum Sanctum tuum ne áuferas a nobis.}
\begin{rubric}
%1928 Rubric:
Then shall follow the Collect(s) for the Day, except when the Communion Service is read; and then the Collect(s) for the Day shall be omitted here.
\end{rubric}
\vspace{-2ex}
\subby{A Collect for Peace}
\lett{O}{God,} who art the author of peace and lover of concord, in knowledge of whom standeth our eternal life, whose service is perfect freedom; Defend us thy humble servants in all assaults of our enemies; that we, surely trusting in thy defence, may not fear the power of any adversaries, through the might of Jesus Christ our Lord. \textit{Amen.}
%\vspace{-1ex}
\subby{A Collect for Grace}
\lett{O}{Lord,} our heavenly Father, Almighty and everlasting God, who hast safely brought us to the beginning of this day; Defend us in the same with thy mighty power; and grant that this day we fall into no sin, neither run into any kind of danger; but that all our doings, being ordered by thy governance, may be righteous in thy sight; through Jesus Christ our Lord. \textit{Amen.}
%Conclusion restored from the Breviary tradition:
%Rubric here already specified in the rubrics page.
%\begin{rubric}
%    The Minister begins with the \emph{Dóminus vobíscum} only if he be in Major Orders. Otherwise, he shall use the second option.
%\end{rubric}
%\vspace{-1ex}
\subby{Conclusion}
%It was difficult to determine whether to use `my' or `our' for those not in Major Orders. Since both are present in the English liturgy, and the beginning uses `our' we decided to remain consistent and use `our'.
\elcol{℣. The Lord be with you.\par
\textit{or,} O Lord, hear our prayer.}{℣. Dóminus vobíscum.\par
\textit{vel,} Dómine, exáudi oratiónem nostram.}
\elcol{℟. And with thy spirit.\par
\textit{or,} And let our cry come unto thee.}{℟. Et cum spíritu tuo.\par
\textit{vel,} Et clamor noster ad te véniat.}

\elcol{℣. Let us bless the Lord (alleluia, alleluia.)}{℣. Benedicámus Dómino (allelúja, allelúja.)}
\elcol{℟. Thanks be to God (alleluia, alleluia.)}{℟. Deo grátias (allelúja, allelúja.)}
\elcol{℣. May the souls {\ding{64}} of the faithful departed, through the mercy of God, rest in peace.}{℣. Fidélium ánim{\ae} {\ding{64}} per misericórdiam Dei requiéscant in pace.}
\elcol{℟. Amen.}{℟. Amen.}
 \begin{rubric}
     The following Prayers shall be omitted here when the Great Litany is said, and may be omitted when the Holy Communion is to follow.
 \end{rubric}
 %The American Tradition experiences a growth of options after the Third Collect. To accommodate this growth, longer than the English Prayer Book, the Minister is given great liberty to remove any of the following prayers. While this may not be the best method to prune an overgrown tree, we do not believe ourselves to be best-suited to choose an alternative.
\begin{rubric}
    And \textsc{Note}, That the Minister may here end the Morning Prayer with such intercessions taken out of this Book, as he shall think fit, or with the Grace.
\end{rubric}
\vspace{-2ex}
\subby{After the Third Collect}
\begin{rubric}
    A Prayer for one's nation (p. \pageref{prayers}) is here said.
\end{rubric}
\vspace{-2ex}
\subbysub{A Prayer for the Clergy and People}
\lett{A}{lmighty} and everlasting God, from whom cometh every good and perfect gift; Send down upon our Bishops, and other Clergy, and upon the Congregations committed to their charge, the healthful Spirit of thy grace; and, that they may truly please thee, pour upon them the continual dew of thy blessing. Grant this, O Lord, for the honour of our Advocate and Mediator, Jesus Christ. \textit{Amen.}
\begin{rubric}
%Move the rubric from Prayers to here:
    Additional Prayers (p. \pageref{prayers}) may be said here.
\end{rubric}
\vspace{-2ex}
\subbysub{A Prayer for all Sorts \& Conditions of Men}
\lett{O}{God,} the Creator and Preserver of all mankind, we humbly beseech thee for all sorts and conditions of men; that thou wouldest be pleased to make thy ways known unto them, thy saving health unto all nations. More especially we pray for thy holy Church universal; that it may be so guided and governed by thy good Spirit, that all who profess and call themselves Christians may be led into the way of truth, and hold the faith in unity of spirit, in the bond of peace, and in righteousness of life. Finally, we commend to thy fatherly goodness all those who are any ways afflicted, or distressed, in mind, body, or estate; (*\margy{*This may be said when any desire the prayers of the Congregation.}especially those for whom our prayers are desired;) that it may please thee to comfort and relieve them, according to their several necessities; giving them patience under their sufferings, and a happy issue out of all their afflictions. And this we beg for Jesus Christ's sake. \textit{Amen.}
\subbysub{A General Thanksgiving}
\begin{rubric}
    The General Thanksgiving may be said by the Congregation with the Minister.
\end{rubric}
\lett{A}{lmighty} God, Father of all mercies, we, thine unworthy servants, do give thee most humble and hearty thanks for all thy goodness and loving-kindness to us and to all men; (*\margy{*This may be said when any desire to return thanks for mercies vouchsafed to them.}to those who desire now to offer up their praises and thanksgivings for thy late mercies vouchsafed unto them.) We bless thee for our creation, preservation, and all the blessings of this life; but above all, for thine inestimable love in the redemption of the world by our Lord Jesus Christ; for the means of grace, and for the hope of glory. And, we beseech thee, give us that due sense of all thy mercies, that our hearts may be unfeignedly thankful; and that we show forth thy praise, not only with our lips, but in our lives, by giving up our selves to thy service, and by walking before thee in holiness and righteousness all our days; through Jesus Christ our Lord, to whom, with thee and the Holy Ghost, be all honour and glory, world without end. \textit{Amen.}
\begin{rubric}
%Move the rubric from Thanksgivings to here:
    The Thanksgivings (p. \pageref{thanksgiving}) may here be offered.
\end{rubric}
\vspace{-2ex}
\subbysub{A Prayer of St. Chrysostom}
\lett{A}{lmighty} God, who hast given us grace at this time with one accord to make our common supplications unto thee; and dost promise that when two or three are gathered together in thy Name thou wilt grant their requests; Fulfil now, O Lord, the desires and petitions of thy servants, as may be most expedient for them; granting us in this world knowledge of thy truth, and in the world to come life everlasting. \textit{Amen.}
\subbysub{The Grace}
\lett{T}{he} grace of our Lord Jesus Christ, {\ding{64}} and the love of God, and the fellowship of the Holy Ghost, be with us all evermore. \textit{Amen.}\par
\begin{center}
    \textsc{Here Endeth the Order of Morning Prayer.}
\end{center}