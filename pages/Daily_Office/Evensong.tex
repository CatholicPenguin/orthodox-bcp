\section{The Order for Daily Evening Prayer}
\fancyhead[RO,LE]{\textit{Evensong}}
\fancyhead[RE,LO]{}
\elcol{℣. O Lord, {\ding{61}} open thou our lips.}{℣. Dómine, {\ding{61}} lábia nostra apéries.}
\elcol{℟. And our mouth shall show forth thy praise.}{℟. Et os nostrum annuntiábit laudem tuam.}
\elcol{℣. O God, {\ding{64}} make speed to save us.}{℣. Deus, {\ding{64}} in adjutórium nostrum inténde.}
\elcol{℟. O Lord, make haste to help us.}{℟. Dómine, ad adjuvándum nos festína.}
\begin{rubric}
    Here, all standing up, the Minister shall say,
\end{rubric}
\elcol{℣. Glory be to the Father, and to the Son, and to the Holy Ghost.}{℣. Glória Patri, et Fílio, * et Spirítui Sancto:}
\elcol{℟. As it was in the beginning, is now, and ever shall be, world without end. Amen.}{℟. Sicut erat in princípio, et nunc, et semper, * et in sǽcula s{\ae}culórum. Amen.}
\elcol{℣. Praise ye the Lord.}{℣. Laudáte Dominum.}
\elcol{℟. The Lord's Name be praised.}{℟. Sit Nomen Dómini Benedíctum.}
\begin{rubric}
%Permission to not say Gloria Patri at the end of each Psalm and Canticle is removed.
    Then shall follow a Portion of the Psalms, according to the Use of this Church. And at the end of every Psalm, and likewise at the end of the \emph{Magnificat}, \emph{Nunc dimittis}, shall be sung or said the \emph{Gloria Patri}, or else the \emph{Gloria in excelsis}, as followeth, %Following rubric added to avoid disconnect between Mass \& Office:
    except during Advent, Septuagesimatide, \& Lent.
\end{rubric}
\vspace{-2ex}
\subby{Gloria in excelsis}
\elcol{\lett{G}{lory} be to God on high, and on earth peace, good will towards men. We praise thee, we bless thee, we worship thee, we glorify thee, we give thanks to thee for thy great glory, O Lord God, heavenly King, God the Father Almighty.\par
    O Lord, the only-begotten Son, Jesus Christ; O Lord God, Lamb of God, Son of the Father, that takest away the sins of the world, have mercy upon us. Thou that takest away the sins of the world, receive our prayer. Thou that sittest at the right hand of God the Father, have mercy upon us.
    For thou only art holy; thou only art the Lord; thou only, O Christ, with the Holy Ghost, art {\ding{64}} most high in the glory of God the Father. Amen.}{\lett{G}{l\smash{ó}ria} in excélsis Deo. Et in terra pax homínibus bon{\ae} voluntátis. Laudámus te. Benedícimus te. Adorámus te. Glorificámus te. Grátias ágimus tibi propter magnam glóriam tuam. Dómine Deus, Rex c{\ae}léstis, Deus Pater omnípotens. \par
    Dómine Fili unigénite, Jesu Christe. Dómine Deus, Agnus Dei, Fílius Patris. Qui tollis peccáta mundi, miserére nobis. Qui tollis peccáta mundi, súscipe deprecatiónem nostram. Qui sedes ad déxteram Patris, miserére nobis. Quóniam tu solus Sanctus. Tu solus Dóminus. Tu solus Altíssimus, Jesu Christe. Cum Sancto Spíritu {\ding{64}} in glória Dei Patris. Amen.}
\begin{rubric}
Then shall be read the First Lesson, according to the Table or Calendar.
\end{rubric}
\vspace{-2ex}
\subby{Magnificat}
\begin{rubric}
%Permission for shorter Office removed.
    After which may be sung or said a Hymn and then shall be sung or said the \emph{Magnificat}, as followeth.
\end{rubric}
\elcol{\lett{M}{y} soul {\ding{64}} doth magnify the Lord, * and my spirit hath rejoiced in God my Saviour.\par
    \secondline{For he hath regarded * the lowliness of his handmaiden.}
    \thirdline{For behold, from henceforth * all generations shall call me blessed.}
    For he that is mighty hath magnified me; * and holy is his Name.\par
    And his mercy is on them that fear him * throughout all generations.\par
    He hath showed strength with his arm; * he hath scattered the proud in the imagination of their hearts.\par
    He hath put down the mighty from their seat, * and hath exalted the humble and meek.\par
    He hath filled the hungry with good things; * and the rich he hath sent empty away.\par
    He remembering his mercy hath holpen his servant Israel; * as he promised to our forefathers, Abraham and his seed, for ever.}
{\lett{M}{agn\smash{í}ficat} {\ding{64}} ánima mea Dóminum.\par
\secondline{Et exsultávit spíritus meus: * in Deo, salutári meo.}
\thirdline{Quia respéxit humilitátem ancíll{\ae} su{\ae}: * ecce enim ex hoc beátam me dicent omnes generatiónes.}
Quia fecit mihi magna qui potens est: * et sanctum nomen ejus.\par
Et misericórdia ejus, a progénie in progénies: * timéntibus eum.\par
Fecit poténtiam in brácchio suo: * dispérsit supérbos mente cordis sui.\par
Depósuit poténtes de sede: * et exaltávit húmiles.\par
Esuriéntes implévit bonis: * et dívites dimísit inánes.\par
Suscépit Israël púerum suum: * recordátus misericórdi{\ae} su{\ae}.\par
Sicut locútus est ad patres nostros: * Ábraham, et sémini ejus in s{\ae}cula.}
\begin{rubric}
Then a Lesson of the New Testament, as it is appointed.
\end{rubric}
\vspace{-2ex}
\subby{Nunc dimittis}
\begin{rubric}
    And after that shall be sung or said the \emph{Nunc dimittis}, as followeth.
\end{rubric}
\elcol{\lett{L}{ord,} {\ding{64}} now lettest thou thy servant depart in peace, * according to thy word.\par
    \secondline{For mine eyes have seen * thy salvation,}
    \thirdline{Which thou hast prepared * before the face of all people;}
    To be a light to lighten the Gentiles, * and to be the glory of thy people Israel.}{\lett{N}{unc} dimíttis {\ding{64}} servum tuum, Dómine, * secúndum verbum tuum in pace:\par
\secondline{Quia vidérunt óculi mei * salutáre tuum,}
\thirdline{Quod parásti * ante fáciem ómnium populórum,}
Lumen ad revelatiónem géntium, * et glóriam plebis tu{\ae} Isra{\"e}l.}

\subby{Apostles' Creed}
\begin{rubric}
    Then shall be said the Apostles' Creed, by the Minister and the People, standing.\par%Permission for alternate words removed.
    \textsc{Note,} the Nicene Creed may be said instead of the Apostles' (p. \pageref{NiceneCreed}).
\end{rubric}
\elcol{\lett{I}{believe} in God the Father Almighty, Maker of heaven and earth:
    And in Jesus Christ his only Son our Lord: Who was conceived by the Holy Ghost, Born of the Virgin Mary: Suffered under Pontius Pilate, Was crucified, dead, and buried: He descended into hell; The third day he rose again from the dead: He ascended into heaven, And sitteth on the right hand of God the Father Almighty: From thence he shall come to judge the quick and the dead.\par
    I believe in the Holy Ghost: The holy Catholic Church; The Communion of Saints: The Forgiveness of sins: The Resurrection of the body: {\ding{64}} And the Life everlasting. Amen.}
    {\lett{C}{redo} in Deum, Patrem omnipoténtem, Creatórem c{\ae}li et terr{\ae}.
     Et in Jesum Christum, Fílium ejus únicum, Dóminum nostrum: qui concéptus est de Spíritu Sancto, natus ex María Vírgine, passus sub Póntio Piláto, crucifíxus, mórtuus, et sepúltus: descéndit ad ínferos; tértia die resurréxit a mórtuis; ascéndit ad c{\ae}los; sedet ad déxteram Dei Patris omnipoténtis: inde ventúrus est judicáre vivos et mórtuos.\par
     Credo in Spíritum Sanctum, sanctam Ecclésiam cathólicam, Sanctórum communiónem, remissiónem peccatórum, carnis {\ding{64}} resurrectiónem, vitam {\ae}térnam. Amen.}

\begin{rubric}
    And after that, these Prayers following, the People devoutly kneeling; the Minister first pronouncing,
\end{rubric}
%\begin{rubric}
%    The Minister begins with the \emph{Dóminus vobíscum} only if he be in Major Orders. Otherwise, he shall use the second option.
%\end{rubric}
\elcol{℣. The Lord be with you.\par
\textit{or,} O Lord, hear our prayer.}{℣. Dóminus vobíscum.\par
\textit{vel,} Dómine, exáudi oratiónem nostram.}
\elcol{℟. And with thy spirit.\par
\textit{or,} And let our cry come unto thee.}{℟. Et cum spíritu tuo.\par
\textit{vel,} Et clamor noster ad te véniat.}
\elcol{℣. Let us pray.}{℣. Orémus.}
\elcol{℣. Lord, have mercy upon us.}{℣. Kýrie, eléison.}
\elcol{℟. Christ, have mercy upon us.}{℟. Christe, eléison.}
\elcol{℣. Lord, have mercy upon us.}{℣. Kýrie, eléison.}
\vspace{-1ex}
\subby{Lord's Prayer}
\begin{rubric}
    Then the Minister, Clerks, and People---all kneeling---shall say the Lord's Prayer with a loud voice.
\end{rubric}
\elcol{\lett{O}{ur} Father, who art in heaven, Hallowed be thy Name. Thy kingdom come. Thy will be done on earth, As it is in heaven. Give us this day our daily bread. And forgive us our trespasses, As we forgive those who trespass against us. And lead us not into temptation; But deliver us from evil. Amen.}
{\lett{P}{ater} noster, qui es in c{\ae}lis, sanctificétur nomen tuum: advéniat regnum tuum: fiat volúntas tua, sicut in c{\ae}lo et in terra. Panem nostrum quotidiánum da nobis hódie: et dimítte nobis débita nostra, sicut et nos dimíttimus debitóribus nostris: et ne nos indúcas in tentatiónem: sed líbera nos a malo. Amen.}
\vspace{-1ex}
\subby{Preces}
\vspace{-1ex}
\begin{rubric}
    Then the Minister standing up shall say
\end{rubric}
\elcol{℣. O Lord, show thy mercy upon us.\par
℟. And grant us thy salvation.\par
℣. O Lord, save the \textit{State}.\par
℟. And mercifully hear us when we call upon thee.\par
℣. Endue thy Ministers with righteousness.\par
℟. And make thy chosen people joyful.\par
℣. O Lord, save thy people.\par
℟. And bless thine inheritance.\par
℣. Give peace in our time, O Lord.\\%MANUAL ADJUSTMENT
\par
℟. For it is thou, Lord, only, that makest us dwell in safety.\par
℣. O God, make clean our hearts within us.\par
℟. And take not thy Holy Spirit from us.}{℣. Osténde nobis, Dómine, misericórdiam tuam.\par
℟. Et salutáre tuum da nobis.\par
℣. Dómine salvam fac \textit{Civitatem}.\par
℟. Et exáudi nos cum invocámus te.\par
℣. Sacerdótes tui induántur Justítia.\par
℟. Et sancti tui exúltent.\par
℣. Salvum fac Pópulum tuum, Dómine.\par
℟. Et bénedic H{\ae}reditáti tu{\ae}.\par
℣. Da pacem Dómine in diébus nostris.\par
℟. Quóniam tu, Dómine, singuláriter in spe constituísti me.\par
℣. Cor mundum crea in nobis, O Deus.\par
℟. Et Spíritum Sanctum tuum ne áuferas a nobis.}
%\elcol{℣. O Lord, show thy mercy upon us.}{℣. Osténde nobis, Dómine, misericórdiam tuam.}
%\elcol{℟. And grant us thy salvation.}{℟. Et salutáre tuum da nobis.}
%\elcol{℣. O Lord, save the \textit{State}.}{℣. Dómine salvam fac \textit{Civitatem}.}
%\elcol{℟. And mercifully hear us when we call upon thee.}{℟. Et exáudi nos cum invocámus te.}
%\elcol{℣. Endue thy Ministers with righteousness.}{℣. Sacerdótes tui induántur Justítia.}
%\elcol{℟. And make thy chosen people joyful.}{℟. Et sancti tui exúltent.}
%\elcol{℣. O Lord, save thy people.}{℣. Salvum fac Pópulum tuum, Dómine.}
%\elcol{℟. And bless thine inheritance.}{℟. Et bénedic H{\ae}reditáti tu{\ae}.}
%\elcol{℣. Give peace in our time, O Lord.}{℣. Da pacem Dómine in diébus nostris.}
%This petition poses a particular problem. Its inclusion (and the inclusion of other petitions) is sporadic in the American tradition. The hesitancy to present a full-throated doctrine of man's sinfulness \& God's justice is manifest here where the English is, as apparent from a cursory glance of the Latin, a perversion. While we are hesitant, cautious, and select when we revert so much from the 1928 into the 1662 or 1549 BCP, the stated principle (to revert when it is a novel departure from the tradition) must be applied to this poor translation even: 
%\elcol{℟. Because there is none other that fighteth for us, but only thou, O God.}{℟. Quia non est álius qui pugnet pro nobis, nisi tu Deus noster.}
%Testing using American version.
%\elcol{℟. For it is thou, Lord, only, that makest us dwell in safety.}{℟. Quóniam tu, Dómine, singuláriter in spe constituísti me.}
%\elcol{℣. O God, make clean our hearts within us.}{℣. Cor mundum crea in nobis, O Deus.}
%\elcol{℟. And take not thy Holy Spirit from us.}{℟. Et Spíritum Sanctum tuum ne áuferas a nobis.}
\begin{rubric}
Then shall be said the Collect(s) of the Day, and after that the Collects and Prayers following.
\end{rubric}
\vspace{-2ex}
\subby{A Collect for Peace}
\lett{O}{God,} from whom all holy desires, all good counsels, and all just works do proceed; Give unto thy servants that peace which the world cannot give; that our hearts may be set to obey thy commandments, and also that by thee, we, being defended from the fear of our enemies, may pass our time in rest and quietness; through the merits of Jesus Christ our Saviour. \textit{Amen.}
 \vspace{-1ex}
\subby{A Collect for Aid against Perils}
\lett{L}{ighten} our darkness, we beseech thee, O Lord; and by thy great mercy defend us from all perils and dangers of this night; for the love of thy only Son, our Saviour, Jesus Christ. \textit{Amen.}
%Conclusion restored from the Breviary tradition:
%\begin{rubric}
%    The Minister begins with the \emph{Dóminus vobíscum} only if he be in Major Orders. Otherwise, he shall use the second option.
%\end{rubric}

\subby{Conclusion}
\elcol{℣. The Lord be with you.\par
\textit{or,} O Lord, hear our prayer.\par
℟. And with thy spirit.\par
\textit{or,} And let our cry come unto thee.\par
℣. Let us bless the Lord (alleluia, alleluia.)\par
℟. Thanks be to God (alleluia, alleluia.)\par
℣. May the souls {\ding{64}} of the faithful departed, through the mercy of God, rest in peace.\par
℟. Amen.}{℣. Dóminus vobíscum.\par
\textit{vel,} Dómine, exáudi oratiónem nostram.\par
℟. Et cum spíritu tuo.\par
\textit{vel,} Et clamor noster ad te véniat.\par
℣. Benedicámus Dómino (allelúja, allelúja.)\par
℟. Deo grátias (allelúja, allelúja.)\par
℣. Fidélium ánim{\ae} {\ding{64}} per misericórdiam Dei requiéscant in pace.\par
℟. Amen.}
%\elcol{℣. The Lord be with you.\par
%\textit{or,} O Lord, hear our prayer.}{℣. Dóminus vobíscum.\par
%\textit{vel,} Dómine, exáudi oratiónem nostram.}
%\elcol{℟. And with thy spirit.\par
%\textit{or,} And let our cry come unto thee.}{℟. Et cum spíritu tuo.\par
%\textit{vel,} Et clamor noster ad te véniat.}
%\elcol{℣. Let us bless the Lord (alleluia, alleluia.)}{℣. Benedicámus Dómino (allelúja, allelúja.)}
%\elcol{℟. Thanks be to God (alleluia, alleluia.)}{℟. Deo grátias (allelúja, allelúja.)}
%\elcol{℣. May the souls {\ding{64}} of the faithful departed, through the mercy of God, rest in peace.}{℣. Fidélium ánim{\ae} {\ding{64}} per misericórdiam Dei requiéscant in pace.}
%\elcol{℟. Amen.}{℟. Amen.}
\begin{rubric}
	{In places where it may be convenient, here followeth the Marian Anthem (p. \pageref{mary}).}
 \end{rubric}
 \vspace{-1ex}
 \begin{rubric}
     The Minister may here end the Evening Prayer with such Prayer, or Prayers, taken out of this Book, as he shall think fit.
 \end{rubric}
% \vspace{-2ex}
\subby{After the Third Collect}
\begin{rubric}
    A Prayer for one's nation (p. \pageref{prayers}) is here said.
\end{rubric}
\vspace{-2ex}
\subbysub{A Prayer for the Clergy and People}
\lett{A}{lmighty} and everlasting God, from whom cometh every good and perfect gift; Send down upon our Bishops, and other Clergy, and upon the Congregations committed to their charge, the healthful Spirit of thy grace; and, that they may truly please thee, pour upon them the continual dew of thy blessing. Grant this, O Lord, for the honour of our Advocate and Mediator, Jesus Christ. \textit{Amen.}
\begin{rubric}
%Move the rubric from Prayers to here:
    Additional Prayers (p. \pageref{prayers}) may be said here.
\end{rubric}
\vspace{-2ex}
\subbysub{A Prayer for all Sorts \& Conditions of Men}
\lett{O}{god,} the Creator and Preserver of all mankind, we humbly beseech thee for all sorts and conditions of men; that thou wouldest be pleased to make thy ways known unto them, thy saving health unto all nations. More especially we pray for thy holy Church universal; that it may be so guided and governed by thy good Spirit, that all who profess and call themselves Christians may be led into the way of truth, and hold the faith in unity of spirit, in the bond of peace, and in righteousness of life. Finally, we commend to thy fatherly goodness all those who are any ways afflicted, or distressed, in mind, body, or estate; (*\margy{*This may be said when any desire the prayers of the Congregation.}especially those for whom our prayers are desired;) that it may please thee to comfort and relieve them, according to their several necessities; giving them patience under their sufferings, and a happy issue out of all their afflictions. And this we beg for Jesus Christ's sake. \textit{Amen.}
\subbysub{A General Thanksgiving}
\begin{rubric}
    The General Thanksgiving may be said by the Congregation with the Minister.
\end{rubric}
\lett{A}{lmighty} God, Father of all mercies, we, thine unworthy servants, do give thee most humble and hearty thanks for all thy goodness and loving-kindness to us and to all men; (*\margy{*This may be said when any desire to return thanks for mercies vouchsafed to them.}to those who desire now to offer up their praises and thanksgivings for thy late mercies vouchsafed unto them.) We bless thee for our creation, preservation, and all the blessings of this life; but above all, for thine inestimable love in the redemption of the world by our Lord Jesus Christ; for the means of grace, and for the hope of glory. And, we beseech thee, give us that due sense of all thy mercies, that our hearts may be unfeignedly thankful; and that we show forth thy praise, not only with our lips, but in our lives, by giving up our selves to thy service, and by walking before thee in holiness and righteousness all our days; through Jesus Christ our Lord, to whom, with thee and the Holy Ghost, be all honour and glory, world without end. \textit{Amen.}
\begin{rubric}
%Move the rubric from Thanksgivings to here:
    The Thanksgivings (p. \pageref{thanksgiving}) may here be offered.
\end{rubric}
\vspace{-2ex}
\subbysub{A Prayer of St. Chrysostom}
\lett{A}{lmighty} God, who hast given us grace at this time with one accord to make our common supplications unto thee; and dost promise that when two or three are gathered together in thy Name thou wilt grant their requests; Fulfil now, O Lord, the desires and petitions of thy servants, as may be most expedient for them; granting us in this world knowledge of thy truth, and in the world to come life everlasting. \textit{Amen.}
\subbysub{The Grace}
\lett{T}{he} grace of our Lord Jesus Christ, {\ding{64}} and the love of God, and the fellowship of the Holy Ghost, be with us all evermore. \textit{Amen.}\par
\begin{center}
    \textsc{Here Endeth the Order of Evening Prayer.}
\end{center}